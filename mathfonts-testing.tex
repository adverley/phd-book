% !TeX spellcheck = en_GB
% !BIB program = biber
% !TEX TS-program = xelatex
% !TEX encoding = UTF-8 Unicode

\documentclass[fontsize=11pt,open=right,numbers=noenddot,]{scrbook} % appendixprefix

% configure directory structure
% MUST be loaded as FIRST package!
\usepackage{subfiles}
\providecommand{\home}{.}

% configure fonts
\usepackage{fontspec}

\usepackage{commath}
\usepackage{amsmath}
\usepackage{amssymb}
\usepackage[geometry]{ifsym}

% Choose a math font
\usepackage[math-style=ISO,warnings-off={mathtools-colon,mathtools-overbracket}]{unicode-math}
\setmathfont{Latin Modern Math}
\setmathfont{XITS Math}
\setmathfont{XITSMath-Regular}
[    Extension = .otf,
      BoldFont = XITSMath-Bold,
]
\DeclareMathAlphabet{\Lmathit}{\encodingdefault}{\familydefault}{m}{it}

\usepackage{textcase}
\usepackage{bm} % for bold in math


\DeclareMathOperator*{\argmax}{\arg\!\max}
\DeclareMathOperator*{\argmin}{\arg\!\min}


\newcommand{\tuple}[1]{\ensuremath{\left \langle #1 \right \rangle }}

\newcommand{\allUppercase}[1]{\expandafter\MakeUppercase\expandafter{#1}}
% alias some commands from isomath

\newcommand{\const}	[1]	{\ensuremath{\MakeUppercase{#1}}}
\newcommand{\var}	[1]	{\ensuremath{\expandafter{#1}}}
\renewcommand{\vec}	[1]	{\ensuremath{\symbfup{#1}}}
\newcommand{\mat}	[1]	{\ensuremath{\MakeUppercase{\symbfup{#1}}}}
\renewcommand{\set}	[1]	{\ensuremath{\mathcal{#1}}}

\DeclareMathOperator{\cloth}{\mathcal{C}}

\newcommand*{\prob}{\mathrm{P}}
\renewcommand*{\captionformat}{\quad}

% FOR reward functions TCN paper part
\DeclareMathOperator*{\euclDist}{dist}
\DeclareMathOperator*{\temporalDist}{temporalDistance}
\DeclareMathOperator*{\tcn}{tcn}

\DeclarePairedDelimiter{\braces}{\{}{\}}
\DeclarePairedDelimiter{\parenthesis}{(}{)}

\newcommand{\frames}{\mathcal{F}}
\newcommand{\anchor}{a}
\newcommand{\positive}{p}
\newcommand{\negative}{n}
\newcommand{\anchorEmbd}{\tcn{(\anchor)}}
\newcommand{\positiveEmbd}{\tcn{(\positive)}}
\newcommand{\negativeEmbd}{\tcn{(\negative)}}
\newcommand{\grad}{\Delta}


\begin{document}
\chapter{Testing playground!} \label{ch:fsds}

Constants: $\const{n}$ and $\const{N}$ and $\const{\nu}$.

Variables: $\var{c}$ and $\var{C}$ and $\var{\theta}$.

Vectors: $\vec{x}$ and $\vec{X}$ and $\vec{\theta}$.

matrices: $\mat{a}$ and $\mat{A}$ and $\mat{\theta}$.

sets: $\set{a}$ and $\set{A}$ and $\set{\theta}$.


The components building up a machine learning system are the dataset, the model, loss function and optimization algorithm.
More formally, we can denote the input data as a set $\set{X}$ consisting of vector $\vec{x}^{(i)} \in \set{X} $ with the superscript $i$ referring to the $i$th observation. In the machine learning domain, this set of predictor variables is called \textit{features}. The set $\set{Y}$ contains the output variables $\var{y}^{(i)} \in \set{Y}$. Concatenating tuples of
$\left\{\left(\vec{x}^{(i)}, \var{y}^{(i)}\right) , i \in 1,\dots,\const{N} \right\}$
, often called \textit{examples}, leads to a dataset which can be used for learning. Central in this learning procedure is the idea of \textit{function approximation} in which a function $f$, parametrized by $\vec{\theta}$, maps an input $\vec{x}^{(i)}$ to its corresponding output $\var{y}^{(i)}$:
\begin{equation*}
	f(\vec{x};\vec{\theta}): \set{X} \mapsto \set{Y}\text{.}
\end{equation*}



\begin{equation*}
	\operatorname{dist}{\left( h(x),h(x_p) \right)} \leq \operatorname{dist}{\left( h(x),h(x_n) \right)} .
\end{equation*}


$f(\vec{p}) = 0$
$f(\vec{x}) = 0$
\end{document}