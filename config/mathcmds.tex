
\DeclareMathOperator*{\argmax}{\arg\!\max}
\DeclareMathOperator*{\argmin}{\arg\!\min}


\newcommand{\tuple}[1]{\ensuremath{\left \langle #1 \right \rangle }}

\newcommand{\allUppercase}[1]{\expandafter\MakeUppercase\expandafter{#1}}
% alias some commands from isomath

\newcommand{\const}	[1]	{\ensuremath{\MakeUppercase{#1}}}
\newcommand{\var}	[1]	{\ensuremath{\expandafter{#1}}}
\renewcommand{\vec}	[1]	{\ensuremath{\symbfup{#1}}}
\newcommand{\mat}	[1]	{\ensuremath{\MakeUppercase{\symbfup{#1}}}}
\renewcommand{\set}	[1]	{\ensuremath{\mathcal{#1}}}

\DeclareMathOperator{\cloth}{\mathcal{C}}

\newcommand*{\prob}{\mathrm{P}}
\renewcommand*{\captionformat}{\quad}

% FOR reward functions TCN paper part
\DeclareMathOperator*{\euclDist}{dist}
\DeclareMathOperator*{\temporalDist}{temporalDistance}
\DeclareMathOperator*{\tcn}{tcn}

% next two commands only work with package mathtools that breaks underbrace
% \DeclarePairedDelimiter{\braces}{\{}{\}}
% \DeclarePairedDelimiter{\parenthesis}{(}{)}

\newcommand{\frames}{\mathcal{F}}
\newcommand{\anchor}{a}
\newcommand{\positive}{p}
\newcommand{\negative}{n}
\newcommand{\anchorEmbd}{\tcn{(\anchor)}}
\newcommand{\positiveEmbd}{\tcn{(\positive)}}
\newcommand{\negativeEmbd}{\tcn{(\negative)}}
\newcommand{\grad}{\Delta}