% !TEX TS-program = xelatex
% !TEX encoding = UTF-8 Unicode

\providecommand{\home}{../..}
\documentclass[\home/main.tex]{subfiles}

\begin{document}

\chapter{Introduction}\label{ch:introduction}

\textbf{High-level problem statement to draw in the reader:}
Robotics promises to relieve humanity from repetitive tasks. Yet, why is there no robot that ties our shoelaces, weeds our gardens, and folds our shirts? 
These tasks break the boundaries of the scripted environment for which existing robot technologies are programmed. Autonomous robots assemble cars and weld miniaturized electrical components at astounding speeds and fine-grained precision. However, industrial robots operate in a cage where every instruction is precisely defined with minimal room for deviation. Nevertheless, when robots are supervised by humans, they can operate outside their safety prisons. For example, the da Vinci\textregistered robot allows surgeons to perform surgery by teleoperating robotic arms through a hand-operated console. However, there exist no adaptive cruise control for surgery yet. 
Evidently, current robotic technologies contain an impressive repertoire of manipulation tasks. However these tasks are solved within industrial or laboratory conditions.  
Gracefully and productivily adopting robots in our daily life will have tremendous impact across industry and society. Robots will be sorting and packing objects in warehouses, installing eletrical wiring in our cars, helping elderly people dress and cleaning-up our houses.
\todo{Missen we hier het belang van dexterous manipulation niet?}
% VOORBEELDEN HIER als volgende zin
%     robot butler: is de was aan het plooien, voor oude mensen aan het zorgen, aan het opruimen, aan het koken 
%     industrie voorbeelden: sort and package objects, bake bread, clean-up the factory, replace eletrical wiring outside the international space station. 
%     probeer de voorbeelden zoveel mogelijk rond manipulatie te houden!


\textbf{Overgang Relevantie deformable objects.}
For the robot butler to become a reality, we need to loosen the assumption of structure in the environment. Nowadays, robotic manipulation solutions assume a fixed structure: the environment is fixed and the configuration of the object does not change. However, the real world presents an infinite supply of configurations.  
Endowing robots with skills to handle objects that deform on interaction helps solving a general problem: how should the robot react if the configuration changes.
% Helpt een probleem op te lossen dat we nu hebben: we lossen robot manipulatie problemen op door te veronderstellen dat er een vaste structuur in het object zelf zit: het vervormt niet. Er zijn echter zoveel vormen van vervormbaarheid en het komt over gans de maatschappij en dagelijks leven voor. 
% --> het lost een algemeen probleem op: wat als de 'configuratie' wijzigt. 

\section{Deformable object manipulation and robotic laundry}
% what is it 
Many day-to-day objects deform upon force interaction. The wires we use for charging our phones, the clothes we carry and the suture used by doctors for stitching wounds. Hence, deformable objects encompasses strings, ropes, cables, clothing, bags and in the extreme case even liquids.  
% why is it relevant?
The vast amount of robotic manipulation focuses on grasping and manipulation of rigid objects. Objects that do not deform siginficantly reduces the input needed for grasping. For example, grasping a blablabla.
On the contrary, deformables require reasoning about their shape and how it deforms depending on the manipulation that is executed.  
However, modern pick-and-place robots already experience difficulties grasping objects that are partially hidden or transparent. We first look at why modern robots and their control algorithms fail. 

\section{Why do traditional control pipelines fail?}
A daily-life example found in common kitchens illustrate the workings of today's industrial robots. The task of slicing food into appropriate dimensions for cooking can be outsourced to kitchen appliances known as food processors\footnote{In the Dutch language, a kitchen robot is unluckily used as a term for kitchen appliances that perform actuated whisk, mix and knead operations.}. Using the food processor requires structuring the environment: one needs to equip the correct blade on the motor shaft, preprocess the food into a manageable size for the blade and feed it through the tube while activating the motor. The consequences of deviating from this setting range from minor failures to dangerous harm. For example, inserting oversized slices of food items might lead to the motor stalling. More dangerously, there is no intelligence in consumer-grade appliances stopping the motor from turning the blades if you decide to insert your hand into the feeding tube.  

Current robotic control pipelines are organized in a similar way as the given food processor example: structuring the environment and decomposing the problem. Decomposition of a large problem into subproblems is a general problem-solving and engineering paradigm that allows making assumptions that simplifies the solution strategy. In robotics, this modular approach is omnipresent and has led to incredible levels of automation and an increase in productivity \autocite{Graetz2018}. However, the innate fear that robots will soon take over our jobs \autocite{cave2019hopes} is unwarranted. 
Open-loop control architectures with no adaptability towards deviations and failures elsewhere in the control pipeline will repeatidly fail when variations occur in the routine. Some of the most advanced robotics research and development teams demonstrate this in the 2015 DARPA Robotics Challenge Finals \autocite{DARPA2015}. The consequences of exposing robots to unstructured environments can be viewed in in published video clips \autocite{darpaVideos}: million-dollar robots tumbling to the ground in, frankly, hilarious ways. In one example, a robot is supposedly turning a valve. However, the robot's walking path is misaligned making it stand next to the valve instead in front it. Consequently, the robot is performing a rotating movement with its arm in thin air while assuming resistance of the valve for balancing. Hence, the robot falls to the ground due to the missing counterbalance. The rotating valve failure brings the inner working of control loops and the consequences to light: a sequence of isolated modules that solve decomposed tasks with no regard for failures down- or upstream in the pipeline. 
This is frappant considering with how much apparent ease humans solve similar tasks. 

\section{Why can data-driven robotics succeed?}
Waarom datadriven robotics 
QUOTE "Historically, researchers have developed ways to manipulate highly deformable objects
that typically use heuristics or geometric methods. For example, Kita et al. [104, 105]
and Maitin-Shepard et al. [136] rely on a heuristic for bimanual fabric manipulation where
one arm lifts a fabric entirely in midair, while the second arm grasps the lowest hanging
corner, detected via corner detection or other geometric techniques. These approaches can
have limited generalization and may require assumptions of the fabric geometry. This thus
motivates the question of whether it is possible to merge recent advances in machine learning
(i.e., deep learning, as we briefly review in Section 1.2) with deformable object manipulation
to reduce assumptions on fabric geometry and to facilitate better generalization."

Hetgene wat verandert --> cfr de figuur van end2end vs modulaire aanpak. 

\section{Waar moet data-driven robotics in verbeteren }
MAAR waarom werkt learning based manipulation nog niet? 

cfr miro board en bijhorende subsecties. Steeds luchtig naar formeel. 

\section{Research outline} 
Our main goal is to investigate how learning-based approaches can be accelerated for . We focus exclusively on the application of robotic manipulation of cloth.

Simulation-driven approach 
Dataset with people folding clothing
Low-cost robot setup to fold cloth in-vivo
Instrumentation
Unsupervised reward function

\section{Publications}

\end{document}
