% !TEX TS-program = xelatex
% !TEX encoding = UTF-8 Unicode

\providecommand{\home}{..}
\documentclass[\home/main.tex]{subfiles}

\begin{document}

\chapter{Samenvatting}

\begin{otherlanguage}{dutch}

Robots met gelijkaardige manipulatievaardigheden als de mens zullen in de toekomst economische welvaart bevorderen, tijd vrijmaken in de samenleving, gevaarlijke handenarbeid overnemen en zorg voorzien aan een vergrijzende bevolking. 
Terwijl robots onze auto's produceren, zijn we nog steeds op onszelf aangewezen om thuis de was te doen.
Deze tekortkoming is te wijten aan de moeilijkheden die we ondervinden bij het omgaan met de variabiliteit die te vinden is in de echte wereld.
Robots die aanwezig zijn in moderne productievestigingen vereisen van ingenieurs dat zij een veilige en voorspelbare omgeving voorzien. Een robot wordt namelijk geprogrammeerd om specifieke bewegingen uit te voeren. Daardoor wordt er verwacht dat objecten steeds op dezelfde locatie en op dezelfde manier aangeleverd worden.
Helaas is het inrichten van een voorspelbare en gestructureerde omgeving ongewenst in dynamische omgevingen waar een breed scala aan objecten wordt verwerkt, vaak in de aanwezigheid van menselijke activiteit. Bijvoorbeeld, moet een menselijke arbeider in een textielbedrijf eerst alle kleren ontvouwen, zodat een robot gemakkelijk de hoekpunten kan vinden en de kledij kan vouwen?
De grote variabiliteit in moderne productieomgevingen en huishoudens vereist immers dat robots voorwerpen verwerken die willekeurige vormen, gewichten en configuraties kunnen aannemen. Deze diversiteit maakt traditionele robotbesturingsalgoritmen en robotgrijpers onbruikbaar om in te zetten in dynamische omgevingen.

Om methodes te ontwikkelen die kunnen omgaan met de hoog veranderlijke aard van menselijke omgevingen, bestuderen we de perceptie en manipulatie van objecten die een oneindige hoeveelheid variaties bieden: vervormbare objecten. Een vervormbaar object of voorwerp verandert van vorm als er een kracht wordt uitgeoefend. Vervormbare objecten zijn alomtegenwoordig in de industrie en de maatschappij: onder andere voedsel, papier, kledij, fruit, kabels en hechtingen. We bestuderen de taak van het automatiseren van het vouwen van kledij. Het vouwen van kledij is een veelvoorkomende huishoudelijke taak die in de toekomst mogelijk door dienstrobots kan worden uitgevoerd. Het manipuleren van kledij is ook relevant in de productie-industrie, waar technisch textiel wordt verwerkt, en in de mode-industrie.

Het omgaan met de vervormbare aard van textiel vereist vooruitgang in zowel hardware als software. 
Op mechanisch vlak moeten actuatoren, scharnieren, sensoren en andere onderdelen ingebouwd worden in de beperkte ruimte van een hand. Daarbij dienen ze zachte materialen te gebruiken die gelijkaardige eigenschappen delen met de menselijke huid. Naast de ontwikkeling van meer capabele handen moeten de besturingsalgoritmen minder assumpties maken over de omgeving waarin robots werken. Het is bijvoorbeeld onrealistisch om te verwachten dat sterk vervormbare voorwerpen zoals textiel zich altijd in dezelfde toestand bevinden om ze door een robot te laten manipuleren.

Omgaan met de variabiliteit in de echte wereld kan gebeuren aan de hand van machinaal leren. Meer concreet is er diep versterkingsgebaseerd leren (RL) dat de functiebenaderingsmogelijkheden van diepe neurale netwerken combineert met het trial-and-error formalisme van \gls{RL}. RL wordt reeds gebruikt om auto's te besturen, met helikopters te vliegen en niet vervormbare voorwerpen te manipuleren. Helaas hebben neurale netwerken heel veel data nodig om patronen te leren herkennen. Deze hoeveelheid aan data is vaak te kostelijk om met een echte robot te genereren.

\keyInText{\emph{Het onderzoek in dit boek behandelt het reduceren van de dataset benodigdheden om systemen te trainen die kledij kunnen herkennen en manipuleren}.
We implementeren een kledijsimulatie om synthetische data te genereren, bouwen een slim textiel dat kan vertellen of het al dan niet gevouwen is, verzamelen een dataset van mensen die kledij plooien, en stellen een systeem voor dat autonoom leert te vertellen hoe goed mensen kledij aan het vouwen zijn.}

% SIMULATION
Een robot aansturen is duur, traag en mogelijks gevaarlijk. Robotici gebruiken daardoor simulatieomgevingen die het gedrag van de robot en de omgeving simuleren. \emph{Er bestaat echter geen robot simulator die compatibel is met bestaande kledijsimulaties}. 
Textielsimulatie wordt gebruikt in de filmindustrie, waar offline berekeningen worden uitgevoerd op vele machines. Er bestaan ook textielsimulaties in videospellen. Deze zijn echter van lage kwaliteit opdat ze real-time kunnen uitgerekend worden. Vanwege het ontbreken van een geschikte kledijsimulatie, schrijven we onze eigen kledijsimulator en integreren we deze in een bestaande robotsimulator. Onze experimenten tonen aan dat we onze simulator kunnen gebruiken om een robot kledij te leren vouwen, en dit binnen 24 uur rekentijd op gewone computationele hardware. 

% SMART CLOTH
Tijdens het autonoom leren met onze kledij- of textielsimulator veronderstellen we dat onze robot de volledige toestand van het textiel kan opvragen. \emph{De toestand van textiel berekenen vraagt echter complexe berekeningen of dure sensortechnologie.} Om deze complexiteit te vermijden, instrumenteren we het textiel door goedkope voelsensoren te integreren. We maken het textiel \textit{slim} door methodes uit het machinaal leren te gebruiken die op basis van de sensordata kunnen vertellen of het doek al dan niet gevouwen is. Vervolgens gebruiken we het slimme textiel om een robot te leren de was te plooien. Onze resultaten tonen aan dat het gebruik van een slim textiel toestaat om op een goedkoop, fysisch robotplatform te leren vouwen.

% DATASET  
Het slimme textiel overbrugt de kloof tussen onze simulatieomgeving en de werkelijke wereld. Het is echter nog steeds nodig om een scalaire waarde uit te rekenen die de taakvooruitgang aangeeft tijdens de uitvoering ervan.
We verwachten dat het beter is om de taakvooruitgang te leren dan ze manueel te specifiëren. Het manueel specifiëren kan zorgen voor menselijke vertekening.  
Er bestaat echter nog geen grote dataset van mensen die kledij vouwen. Omwille van deze reden verzamelen we onze eigen dataset. Onze dataset bevat ruwweg \qty{300000}{} RGB-D beelden die geannoteerd zijn met de positie van de armen, kwaliteitslabels en tijdstempels die de deelstappen aanduiden. Deze dataset kan worden gebruikt voor onderzoek naar actieherkenning en het versnellen van leren met gebruik van voorbeelden.

% REWARD FUNCTIONS
Het is vereist om de bewegingen van de menselijke voorbeelden in onze dataset te vertalen naar de morfologie van de robot. 

Om te vermijden dat deze vertaalslag leidt tot het blind na-apen van bewegingen, geven we er de voorkeur aan om een beloningssignaal te leren in plaats van de handelingen zelf.
Er bestaat echter nog geen methode die een beloningssignaal kan leren, althans niet zonder simultaan ook het gewenste gedrag te leren. Deze koppeling zorgt ervoor dat alle problemen die geassocieerd worden met versterkingsgebaseerd leren worden overgenomen bij het leren van beloningen. 
We introduceren een methode die het leren van beloningen en gedrag ontkoppelt door rechtstreeks het taakverloop te voorspellen op basis van camerabeelden van menselijke voorbeelden. Aangezien onze methode niet vereist om individuele beelden te voorzien van een vooruitgangsgetal, vermijden we het insijpelen van menselijke vertekening. Onze methode toont het eerste systeem dat instaat is om de taakvooruitgang uit te drukken van mensen die kledij vouwen. 

% METtaaaaaaa
De robotbutler is nog niet voor morgen. Robots moeten eerst leren werken in dynamische omgevingen. Hiervoor moet toekomstig onderzoek zich toespitsen op het integreren van software en hardware. In ons onderzoek hebben we de kracht hiervan aangetoond door middel van een slim textiel dat de robot vertelt hoe het gevouwen is. Zo vermijden we dat de robot zelf de toestand van het textiel moet leren. Door deze eigenschap kan de robot binnen de dag leren om het textiel te vouwen.
Het verder doordrijven van deze geïntegreerde visie op hardware en software leidt tot intelligentie die ingebouwd is in het lichaam zelf. Het simultaan optimaliseren van het lichaam en het brein staat toe om robotgrijpers te ontwikkelen die aangepast zijn om specifieke taken uit te voeren. Robots kunnen deze grijpers gebruiken om representaties te leren die toestaan om te begrijpen hoe de wereld werkt. Zo zullen robots de gevolgen van hun acties leren begrijpen en uiteindelijk taken leren uitvoeren door te interageren met hun omgeving, te kijken naar hoe mensen de taak uitvoeren en door feedback van geïnstrumenteerde objecten op te vragen. 
Dit holistisch proces zal de toekomstige robotbutler toestaan om mensen te begrijpen en een breed gamma aan taken te leren uitvoeren. 

\end{otherlanguage}

\end{document}
