% !TEX TS-program = xelatex
% !TEX encoding = UTF-8 Unicode

\providecommand{\home}{..}
\documentclass[\home/main.tex]{subfiles}

\begin{document}

\chapter{Samenvatting}

\begin{otherlanguage}{dutch}

Robots met gelijkaardige manipulatievaardigheden als de mens zal economische welvaart bevorderen, tijd vrij maken in de samenleving, gevaarlijkse handenarbeid overnemen en zorg voorzien aan een vergrijzende bevolking. 
Terwijl robots onze auto's produceren, zijn we nog steeds op onszelf toegewezen om thuis de was te doen.
Deze tekortkoming is te wijten aan de moeilijkheden bij het omgaan met de variabiliteit die te vinden is in de echte wereld.
Robots in moderne productievestigingen vereisen van ingenieurs dat zij een veilige en voorspelbare omgeving voorzien. Een robot wordt namelijk geprogrammeerd om specifieke bewegingen uit te voeren. Daardoor wordt er verwacht dat objecten steeds op dezelfde locatie en op dezelfde manier aangeleverd worden.
Helaas is het inrichten van een voorspelbare en gestructureerde omgeving ongewenst in dynamische omgevingen waar een breed scala aan objecten wordt verwerkt, vaak in de aanwezigheid van menselijke activiteit. Bijvoorbeeld, moet een menselijke arbeider in een textielbedrijf eerst alle kleren ontvouwen, zodat een robot gemakkelijk de hoekpunten kan vinden en het kledij kan vouwen?
De grote variabiliteit in moderne productieomgevingen en huishoudens vereist immers dat robots voorwerpen verwerken die willekeurige vormen, gewichten en configuraties kunnen aannemen. Deze diversiteit maakt traditionele robotbesturingsalgoritmen en robotgrijpers onbruikbaar voor gebruik in dynamische omgevingen.

Om manieren te vinden die kunnen omgaan met de hoogveranderlijke aard van menselijke omgevingen, bestuderen we de perceptie en manipulatie van objecten die een oneindige hoeveelheid variaties bieden: vervormbare objecten. Een vervormbaar voorwerp verandert van vorm als er een kracht wordt uitgeoefend. Vervormbare objecten zijn alomtegenwoordig in de industrie en de maatschappij: onder andere voedsel, papier, kleren, fruit, kabels en hechtingen. We bestuderen de taak van het automatiseren van het vouwen van kledij. Het vouwen van kleding is een veelvoorkomende huishoudelijke taak die in de toekomst mogelijk door dienstrobots zal worden uitgevoerd. Het manipuleren van kleding is ook relevant in de productieindustrie, waar technisch textiel wordt verwerkt, en in de mode-industrie.

Het omgaan met de vervormbare aard van textiel vereist vooruitgang in zowel hardware als software. 
Op mechanisch vlak moet actuatoren, onderdelen, scharnieren en sensoren in de beperkte ruimte van een hand inbouwen en daarbij zachte materialen gebruiken die gelijkaardige eigenschappen delen met de menselijke huid. Naast de ontwikkeling van meer capabele handen moeten de besturingsalgoritmen assumpties loslaten over de omgeving waarin robots opereren. Het is bijvoorbeeld onhaalbaar om te verwachten dat sterk vervormbare voorwerpen zoals textiel zich altijd in dezelfde configuratie bevinden om ze door een robot te laten manipuleren.

Omgaan met de variabiliteit in de echte wereld kan gebeuren met machinaal leren. Meer concreet is er diep versterkingsgebasseerd leren (RL) dat de functiebenaderingsmogelijkheden van diepe neurale netwerken combineert met het trial-and-error formalisme van \gls{RL}. RL is reeds gebruikt om auto's te besturen, met helikopters te vliegen en stijve voorwerpen te manipuleren. Helaas hebben neurale netwerken heel veel data nodig om patronen te leren herkennen. Deze hoeveelheid aan data is vaak te kostelijk om met een echte robot te genereren.

\emph{Het onderzoek uiteengezet in dit boek behandelt het reduceren van de dataset benodigdheden om systemen te trainen dat kledij kunnen herkennen en manipuleren}.
We implementeren een kledijsimulatie om synthetische data te genereren, bouwen een slim textiel dat kan vertellen of het al dan niet gevouwen is, verzamelen een dataset van mensen die kledij plooien, en stellen een systeem voor dat autonoom leert te vertellen hoe goed mensen kledij aan het vouwen zijn.

% SIMULATION
Een robot aansturen is duur, traag en mogelijks gevaarlijk. Robotici gebruiken daardoor simulatieomgevingen dat het gedrag van de robot en omgeving simuleren. \emph{Er bestaat echter geen robot simulator dat integreert met bestaande kledijsimulaties}. Textielsimulatie wordt gebruikt in de filmindustrie, waar er offline de berekeningen op vele machines worden uitgevoerd. Er bestaat ook textielsimulatie in videospellen. Deze zijn echter van lage kwaliteit zodat ze real-time kunnen uitgerekend worden. Vanwege het ontbreken van een geschikte kledijsimulatie, schrijven we onze eigen kledijsimulator en integreren we deze in een bestaande robotsimulator. Onze experimenten tonen dat we onze simulator kunnen gebruiken om een robot te leren kledij te vouwen binnen 24 uur rekentijd op gewone computationele hardware. 

% SMART CLOTH
Tijdens het autonoom leren met onze textielsimulatie veronderstellen dat onze robot de volledige toestand van het doek kan opvragen. \emph{De toestand van textiel berekenen vraagt echter complexe berekeningen of dure sensortechnologie.} Om deze complexiteit te vermijden, integreren we goedkope voelsensoren in het doek. We maken het doek \textit{slim} door machine learning methodes te gebruiken dat op basis van de sensordata kan vertellen of het doek al dan niet gevouwen is. Vervolgens gebruiken we het slimme textiel om een robot leren de was te plooien. Onze resultaten tonen aan dat het gebruik van een slim textiel toestaat om op een goedkoop, fysisch robotplatform te leren vouwen.

% DATASET  
Het slimme textiel overbrugt de kloof tussen onze simulatieomgeving en de werkelijke wereld. Het is echter nog steeds nodig om een scalaire waarde uit te rekenen dat de taakvooruitgang aangeeft tijdens het uitvoeren van de taak. We verwachten dat het leren van deze taakvooruitgang beter geschikt om menselijke vooroordelen te vermijden in het geval van het manueel specifieren van taakbeloningen. Er bestaat echter nog geen grote dataset met mensen die kledij vouwen. Vanwege deze reden verzamelen we onze eigen dataset. Onze dataset bevat ruwweg \qty{300000}{} RGB-D beelden die geannoteerd zijn met de positie van de armen, kwaliteitslabels en tijdstempels die de substappen aanduiden. Deze dataset kan worden gebruikt voor onderzoek naar actieherkenning en het versnellen van leren met gebruik van voorbeelden.

% REWARD FUNCTIONS
Het is vereist om de bewegingen van de menselijke voorbeelden in onze dataset te vertalen naar de morfologie van de robot. Om te vermijden dat deze vertaalslag leidt tot het blind naapen van bewegingen, geven we de voorkeur om de een beloningssignaal te leren uit de data in plaats van het benodigde gedrag. 
Er bestaat echter nog geen methode dat een beloningssignaal kan leren, althans niet zonder simultaan ook het gewenste gedrag te leren. Deze koppeling zorgt ervoor dat alle problemen die geassocieerd worden met versterkingsgebasseerd leren worden overgenomen bij het leren van beloningen. 
We introduceren een methode dat het leren van beloningen en gedrag ontkoppelt door rechtstreeks het taakverloop te voorspellen op basis van camerabeelden van menselijke voorbeelden. Aangezien onze methode niet vereist om indiviudele beelden te voorzien van een vooruitgangsgetal, vermijden we het insijpelen van menselijke vooroordelen. Onze methode is het eerste systeem dat instaat is om taakvooruitgang uit te drukken van mensen dat kledij vouwen. 

% METtaaaaaaa
De robotbutler is nog niet voor morgen. Robots moeten eerst leren werken in dynamische omgevingen. Hiervoor moet toekomstig onderzoek zich toespitsen op het integereren van software en hardware. In ons onderzoek hebben we de kracht hiervan aangetoond door middel van een slim textiel dat de robot vertelt hoe het gevouwen is. Zo vermijden we dat de robot zelf de toestand van het textiel moet leren. Vanwege deze eigenschap kan de robot binnen de dag leren om het textiel te vouwen.
Het verder doordrijven van deze geïntegreerde visie op hardware en software leidt tot intelligentie dat ingebouwd is in het lichaam zelf. Het simultaan optimalizeren van het lichaam en het brein staat toe om robotgrijpers te ontwikkelen dat aangepast zijn om specifieke taken uit te voeren. Robots kunnen deze grijpers gebruiken om representaties te leren dat toestaan te begrijpen hoe de wereld werkt. Zo zullen robots de gevolgen van hun acties leren begrijpen en uiteindelijk taken leren uitvoeren door te intrageren met de omgeving, te kijken naar hoe mensen de taak uitvoeren en feedback van geinstrumenteerde objecten op te vragen. 
Dit holistisch proces zal de toekomstige robotbutler toestaan om mensen te begrijpen en een breed gamma aan taken te leren uitvoeren. 

\end{otherlanguage}

\end{document}
