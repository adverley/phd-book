% !TEX TS-program = xelatex
% !TEX encoding = UTF-8 Unicode

\providecommand{\home}{../..}
\documentclass[\home/main.tex]{subfiles}

\begin{document}

\chapter{Robotic folding in simulation}\label{ch:simulation}

In this chapter, we explore the use of simulation for training robotic controllers for deformable object manipulation. We frame the usage and difference of robots and physics simulators, provide a comparison between simulation technologies and implementations, and finally discuss results on learning to fold in simulation. 

\section{Digital twins}


% Why use simulation: rise of virtual env "digital twin"

% Robot vs cloth simulation

% What do we need: our use-case 

\section{Robotic simulation {\tiny pybullet, gazebo, nvidia, v-rep, unity. Nadruk op Unity en motivatie.}}

% What is a robot simulator and how does it differ from physics simulator. Also real integrations: URDF, ROS communication with real robot. 

% Short comparison of robot simulation technologies

% why do we go for Unity

\section{Cloth simulation \tiny{particle system, msd, integrator , ...} }
% OP REIS TODO: schrijf particle simulation gedeelte. Lees eerst de typische papers voor structuur maar gebruik uiteindelijk muller zijn notes. 

%Zie ook "Robotic manipulation and sensing of deformable objects in domestic and industrial applications: a survey" p4 voor overzicht om te introduceren maar focus op particle based methods.
%  zie ook Dexterous Robotic Manipulation of Deformable Objects with Multi-Sensory Feedback - a Review van khalil
% background sectie 3 van SoftGym": https://arxiv.org/pdf/2011.07215.pdf
%  Voor de specifieke methode, zie ook p47 van Seita zijn boek 
%  Zie ook Muller notes about deformable object manipulation p11



\section{Learning to fold in simulation}
% hier moet ik nog wat nadenken over het doel van deze sectie 
% het demonstreren dat de simulator werkt, dat plooien kan 
% tonen dat DRL werkt maar veel iteraties nodig heeft. De reward functie is godmode en moeilijk te transfereren naar het echt 
% eventueel een behavioral cloning approach ook tonen. Dan showcasen waar het zich vastrijdt
% eventueel iets over sim2real problemen dat hier al naar boven komen 
\subsection{Deep reinforcement learning setup for cloth folding in simulation}
\subsection{Results \tiny{om te tonen dat leren mogelijk is in deze setup maar wat de moeilijkheden zijn: de reward, godmode en aantal obs}}

\footnote{The implementation of the simulation is developed in context of a software engineering course with students }


\section{Conclusion {\tiny considerations, approaches, our approach, DRL works but reward function godmode and many iterations}}

\end{document}