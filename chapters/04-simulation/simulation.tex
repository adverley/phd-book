% !TEX TS-program = xelatex
% !TEX encoding = UTF-8 Unicode

\providecommand{\home}{../..}
\documentclass[\home/main.tex]{subfiles}

\begin{document}

\chapter{Robotic folding in simulation}\label{ch:simulation}
\section{Digital twins {\tiny verschil robot en cloth simulaties schetsen. Onze specifieke use-case duidelijk maken. }}
\section{Robotic simulation {\tiny pybullet, gazebo, nvidia, v-rep, unity. Nadruk op Unity en motivatie.}}
\section{Cloth simulation \tiny{particle system, msd, integrator , ...} }

\section{Learning to fold in simulation}
% hier moet ik nog wat nadenken over het doel van deze sectie 
% het demonstreren dat de simulator werkt, dat plooien kan 
% tonen dat DRL werkt maar veel iteraties nodig heeft. De reward functie is godmode en moeilijk te transfereren naar het echt 
% eventueel een behavioral cloning approach ook tonen. Dan showcasen waar het zich vastrijdt
% eventueel iets over sim2real problemen dat hier al naar boven komen 
\subsection{Deep reinforcement learning setup for cloth folding in simulation}
\subsection{Results \tiny{om te tonen dat leren mogelijk is in deze setup maar wat de moeilijkheden zijn: de reward, godmode en aantal obs}}
\section{Conclusion {\tiny considerations, approaches, our approach, DRL works but reward function godmode and many iterations}}

\end{document}