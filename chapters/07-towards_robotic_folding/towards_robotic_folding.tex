% !TEX TS-program = xelatex
% !TEX encoding = UTF-8 Unicode

\providecommand{\home}{../..}
\documentclass[\home/main.tex]{subfiles}

\begin{document}

\chapter{Towards robotic manipulation of clothing }\label{ch:towards_robotic_folding}
\section{cfr structure literature}

Too much end2end, there is a lot of prior work exposing good tricks: use gravity to unfold, immobilize parts of the cloth, make dedicated grippers ... cfr first part of literature. 

Curricilum!
Sensing!


Multimodal learning.
Unify RL and imitation learning. p14 a survey of deep network solutiosn for learning control in roboits: from RL to imitation.

Datasets 
    Self-supervision 
        labeled cloth 
    In the wild 
    

Structure:
\begin{easylist}
    & Modeling and simulation
    & Perception 
    & Control 
    & Hardware 
        && Sensors, tactile 
        && end-effectors 
\end{easylist}
The most promising research directions point toward addressing multimodal perception, integration of analytical and data-driven methods, and development and use of simulators for data generation, model evaluation, and benchmarking.
\end{document}