% !TEX TS-program = xelatex
% !TEX encoding = UTF-8 Unicode

\providecommand{\home}{../..}
\documentclass[\home/main.tex]{subfiles}

\begin{document}

% TODO: check ook nog laatste secties van [2018 IJRR]The limits and potentials of deeplearning for robotics.pdf

\chapter{Towards robotic manipulation of clothing}\label{ch:towards_robotic_folding}

In this research dissertation, we considered the problem of learning robotic folding of clothing items by using simulation, smart clothing and learning task progression from human demonstrations. In this chapter, we zoom-out to take a birds-eye perspective on the field of robotic folding in order to highlight potential areas of research. Our goal is to describe high-potential areas of research that, according to the research done in this dissertation, 

\section{Improving sample efficiency}



Quote van andere paper: there is a spectrum, rather than a dichotomy, between programming and data.

\section{Datasets}

\section{Simulation}

\section{Sim2Real}

\section{Sensing and representation}

\section{Hardware for robotic folding}

\section{To categorize delete me later}

\end{document}