% !TEX TS-program = xelatex
% !TEX encoding = UTF-8 Unicode

\providecommand{\home}{../..}
\documentclass[\home/main.tex]{subfiles}

\begin{document}

\chapter{Towards robotic manipulation of clothing }\label{ch:towards_robotic_folding}
\section{cfr structure literature}

Too much end2end, there is a lot of prior work exposing good tricks: use gravity to unfold, immobilize parts of the cloth, make dedicated grippers ... cfr first part of literature. 

Curricilum!  spectrum: Manual, heuristic or learned
    % https://arxiv.org/pdf/2109.11978.pdf  --> 

        %  --> locomotion met PPO. Twee trucken: simulatie massaal in parallel op GPU (IsaacGym’s PhysX engine) en heuristisch curriculum.
        %  Heuristisch curr is langzaam de helling van het terrein en trappen verhogen. De heuristiek houdt ook rekening met forgetting door af en toe een random lager level te introduceren of het algemene level terug te verlagen.

Sensing!


Multimodal learning.
Unify RL and imitation learning. p14 a survey of deep network solutiosn for learning control in roboits: from RL to imitation.

Datasets 
    Self-supervision 
        labeled cloth 
    In the wild zoals youtube
    
    

Structure:
\begin{easylist}
    & Modeling and simulation
    & Perception 
    & Control 
    & Hardware 
        && Sensors, tactile 
        && end-effectors 
\end{easylist}
The most promising research directions point toward addressing multimodal perception, integration of analytical and data-driven methods, and development and use of simulators for data generation, model evaluation, and benchmarking.

DATASETS:

Datasets have enabled deep learning. Datasets have enabled great successes for robotic manipulation. In the case of grasping objects, it seems Dex-Net takes the lead because it allows to completely frame the problem as a supervised learning problem leading to a more qualitative learning signal. Additionally, working in simulation has allowed them to generate $6.7$ million images which is approximately $7$ times the size of the largest data collection setup of a single institution~\autocite{Levine2016}. However, synthetic data is plagued with transferability problems which makes realistic simulations, robust algorithms and system identification crucial. 
We have good retail fashion datasets~\autocite{DeepFashion, DeepFashion2, FashionAI}, but now we need more modalities, interactions and need to find a way to make them useful for robotic manipulation tasks. 
I really like the FashionAI AttributeNet that disentangles attributes using experts, labels some data, train a network and stop training when the network is failing on some attributes. This allows scaling up data collection.
Existing robotic interaction datasets only contain vision, no touch. 
Existing robotic learning datasets are too versplinderd, no common standard, hard to consolditate. 
We need high quality and high volume datasets like the retail have, but then for robotic applications. 

===================
SIMULATORS 
===================
Deformable bodies require reasoning about the shape, dynamics and material properties of the object. -> this is why NN are so interesting. They can learn it.

Embed simulation structure in neural netwokr. 

MASSAAAL IN PARALLELL. GPUS are the way!

interdisciplinary: medical img -> deformable object modeling: shape representation. zie comment hieronder.
    % Sun et al. used methods of particle-based smoothing as well as non-parametric belief propagation on a loopy graphical model capturing the temporal periodicity of the heart, with the objective being to estimate the current state of the object not only from the data observed at that instant but also from predictions based onpastandfutureboundaryestimates. Eventhoughwedid not find examples of this method being used in robotics, it seems a suitable candidate since it models the shape change through time implicitly and would thus allow the robot to keep track of the evolving shape of an object during manipulation. 


diff physics:
    optimize the following:
        control
        system identification  (sim2real gap overcomming)
        body 


        % (from: https://www.frontiersin.org/articles/10.3389/frobt.2020.00082/full)
        DL dynamics models 
        Deep NN models are similar to      the NN dynamics models described in section 3.2.3 but are more
        scalable; for example, they can be used to model interactions
        between multiple objects in a single scene. We call these
        methods deep learning-based dynamics models. An advantage
        of these models is that they can learn directly from sensory
        observations (e.g., images of the observed scene) with high
        accuracy by virtue of their end-to-end differentiability, unlike
        the more computationally expensive analytical dynamics models
        in section 3, which are mostly too complex to differentiate.
        However, deep learning-based models have two limitations. First,
        most methods based on differentiable dynamics models are tested
        on limited scenarios with a few interacting rigid objects (e.g.,
        balls colliding), which makes them difficult to generalize to
        more complicated behaviors of deformable objects. There are
        differentiable dynamics models that can learn more complex
        deformable object behavior, as in Mrowca et al. (2018) and Li
        et al. (2018); however, they may exhibit visually implausible
        deformations, such as the loss of shape preservation over time,
        or may not be able to deal with visual perception that includes
        partial or noisy observations of the state of objects, for instance
        due to the occlusion of a robot manipulator with an object.
        Furthermore, NN-based methods mostly work on entities of fixed
        dimensionality and do not exploit characteristics, such as the
        flexibility of splines.
        In conclusion, for the future we need algorithms that
        can deal with various challenging manipulation scenarios
        where previously unseen objects appear in a scene (e.g.,
        in contexts, such as cooking or nursing) by incrementally
        learning the shape and dynamics of a previously unseen object
        simultaneously while planning and controlling manipulation.
        To develop such methods, different approaches, such as
        integrating efficient differentiable models with physically
        accurate dynamics models, should be investigated. Also, models
        of dynamics should be adapted to work on different families of
        shape representations.
\end{document}