% !TEX TS-program = xelatex
% !TEX encoding = UTF-8 Unicode

\providecommand{\home}{../..}
\documentclass[\home/main.tex]{subfiles}

\begin{document}

\chapter{Towards robotic manipulation of clothing }\label{ch:towards_robotic_folding}


==============
HARDWARE:
=============

In food, we use pneumatic gripper because they solve the grasping problem realitively easy. However, manipulation tasks are difficult with pneumatic grippers. For example, grasping a sliced potato and putting it with the chopped side in a confined space is difficult to execute with pneumatic end-effectors. Hence we need dexteours hands, with fingers , sensors, small actuartors, precision. 

coevolution of gripper and task. Cfr Dries, leapfrog. 
Can give rise to designing beyond the layout and constrainst of the human hand: for example having two opposable thumbs. %(cfr Billard2019). 

Dexterous manipulation requires dexterity in the materials of a robotic hand. Today, most robotic hands are composed of rigid components leading to a stiff and rigid design. % Zie ook Billard2019 p4 links kolom --> vandaag zijn robotic hands stijf door motoren en links. 

Bimanual robot manipulation is underdeveloped \autocite{Billard2019}. Important for cloth: cloth folding in general requires the use of two manipulators. For example, fix the cloth against the tables while using the other hand to fold it. 

Further research toward several technologies vital for meeting and exceeding human dexterity and fine manipulation capabilities is needed .


 The community is uncertain about what types of end effectors are needed for optimally grasping and constraining cloth-like objects. While there are suggestions for cloth-specific grippers, there is no study benchmarking their usefulness and comparative performance. In addition, the question can be raised whether a future robot should have a general-purpose gripper or is able to quickly switch to  task-specific implements.  

 Sensing is recently identified as the most important area of research in a recent technical workshop on the subject of deformable object manipulation \autocite{zhu:hal-02980281}. RGB, depth, tactile, ... 


--------------------------
    reward learning
--------------------------
Learning rewards is equivalent to learning which signal to give to an agent such that its personal performance on the task can increase. This constitutes two elements: personal and performance. We focus on the performance part: what is task intent being encoded in the demonstrations provided by the demonstrator? However, it might be beneficial to couple the learning process to the learning characteristics of the agent. One method to incorporate this characteristic is curriculum learning. In curriculum learning, we adapt the setting of the task to the performance level and characteristics of the learning agent.  


 ----------------------------------------------------
    Human-robot collaboration
----------------------------------------------------
An aspect untouched in this research but of great importance in DOM: collaborative manipulation of deformable objects between robot and human. Examples include robot-assisted dressing \autocite{Gao2016}, assitive support \autocite{lu2017human} and bed-sheet folding \autocite{Kruse2015}.


-----------------
UNCATEGORIZED
----------------- 

% Category planning? 
The main difficulties of robotic cloth folding arise from the infinite configuration space of cloth. In our work, we tackled this by finding a compact representation from human demonstrations. An alternative approach would be to purposefully constrain the configuration space. This idea is recently coined in deformable linear objects \autocite{Zhu2020}. The same principle of constraining the object can be applied to cloth folding. For example, the robot should purposefully pinpoint parts of the cloth for folding. Such idea is already being applied when using a table for folding. For example, swinging up the cloth and releasing it on the table reduces the configuration space to the plane spanned by the table. Similarly, a robot can exploit regions less suscuptible to deformations when the mass is not uniformly distributed in the cloth. Such constraints simplifies the planning. 

Dynamic control: for example swing up cloth to unfold and remove wrinkles. Is both a difficult planning (complex dynamics) and control (force control) problem.


Safety: folding large bedsheets can be in assistance with human operator. Embed learning agents into physical environments with humans. 



\end{document}