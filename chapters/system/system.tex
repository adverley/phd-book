% !TEX TS-program = xelatex
% !TEX encoding = UTF-8 Unicode

\providecommand{\home}{../..}
\documentclass[\home/main.tex]{subfiles}

\begin{document}

\chapter{System Architecture and Specifications}\label{system}

In this \namecref{system}, we discuss the set of specifications that the system should meet. These consist of measurement related specifications and electrical specifications. In the second part of this \namecref{system}, the system architecture is highlighted and compared to alternatives. In the subsequent \namecrefs{system}, each of the components from this architecture will be implemented and discussed. 

\section{System Specifications}\label{system:specs}

  Before starting the design, a set of specifications is needed. These are mentioned in this \namecref{system:specs}. We will start with the measurement related specifications, followed by the electrical ones. There are no real optical specifications, since the optical components (laser and photodiode) are fixed beforehand. 
  
  The system should have sufficient accuracy such that it can measure ambient levels of \acrshort{CO2} and dangerous levels if \acrshort{CO}. Ambient \acrshort{CO2} levels are around \SI{410}{ppm} \cite{CO2-levels} and the maximum allowed daily dose of \acrshort{CO} is in the range of \SIrange{35}{200}{ppm} continuous exposure per day \cite{CO-levels}. As a consequence, we require that the system is able the measure concentrations with an uncertainty of up to \SI{\pm 10}{ppm}.
  
  These measurements should be performed in real-time, though a single measurement need not have an accuracy of \SI{\pm 10}{ppm}, a combination of them should. Measurements can be combined by averaging them to decrease the noise. Since measurements are assumed to be performed in real-time, we require that \SI{\pm 10}{ppm} accurate measurements are performed within \SI{1}{\second}. 
  
  As discussed in \cref{background:WMS}, the system needs to capture two signals at different frequencies at the detector side. We require that the gain difference between both signals is at most \SI{0.4}{\deci\bel}, since this will directly result in concentration imprecision. During system design, we will work mostly with components that have a \SI{5}{\percent} tolerance, which justifies this \SI{0.4}{\deci\bel} offset gain offset (also approximately \SI{5}{\percent}). We also require that the entire system is able to run at a low supply voltage (e.g.\ \SI{5}{\volt}). This allows to scale the system more easily for integrated applications. 

\section{System Architecture}
  
  Prior to starting the system design, we must fix a system architecture. A general block diagram is provided in \cref{system:architecture_block_diagram}. We will now discuss why this architecture was selected and how it enables us to measure gas concentrations.
  
  The architecture from \cref{system:architecture_block_diagram} can be subdivided into three parts: the analogue/digital electronics (green blocks), optics (white blocks) and the microcontroller which implements several features in its programme code (blue blocks). This dissertation focusses on the electronics and digital part, a laser and photodetector were provided. The gas cell may vary depending on environment. %We did not design these  optical components. As a consequence, a separate chapter is not required.
  
  The operation of this architecture is straightforward. We start from the microcontroller output (upper left) part of the diagram that controls the \acrfull{DAC}. The  \acrshort{DAC} generates a voltage signal consisting of a bias component, line scanning ramp (or triangle) and high frequency sine modulation. These are voltage signals, while lasers are usually current controlled. The voltage controlled current source ($V \rightarrow I \ \text{conv.}$) converts this voltage signal into a current signal that drives the laser. 
  
  The emitted laser line is sent to the gas cell and some of this light is incident on a \acrfull{PD}. A current signal is generated depending on the amount of incident light on the active area of the \acrshort{PD}. This (typically small) current is converted into a voltage and amplified by the \acrfull{TIA}. The first and second order harmonics are the signals of interest (see \cref{background:setting-and-background}), not the slowly varying ramp (or triangle) signal nor the bias level. The varying ramp (or triangle) and bias signal are filtered out using a high pass filter. The resulting signal can possibly be amplified using a \acrfull{VGA} that is controlled by the microcontroller. Before converting the analogue signal into a digital signal, low pass filtering is performed. This is done to remove spurious components due to interference and to prevent aliasing. 
  
  The (external) \acrfull{ADC} converts the analogue signal into a digital signal at a fixed sampling rate. This signal is then passed to the \acrshortpl{LIA}. These extract the desired information (the second order harmonic $2f$) and reference signal (the first order harmonic $1f$) that are combined into a ratio ($R$). A model is then fitted to this ratio to extract the concentration.
  
  The \acrshortpl{LIA} will be implemented on the Explorer 16 Development Board with 100-pin \acrshort{PIM} (ref.\ number: DM240001), a prototyping board from Microchip featuring the dsPIC33FJ256GP710A microcontroller. This microcontroller operates at \SI{40}{MIPS} (\acrlong{MIPS}) and features optimised \acrshort{DSP} routines and instructions. Additionally, it also includes a \SI{12}{bit} \acrshort{ADC}. An external \SI{16}{bit} \acrshort{ADC} was selected though, since these can provide more accurate analogue-to-digital conversion. This us allows to better detect signals near the \SI{10}{ppm} accuracy limit. Furthermore, the signal does also not need to be amplified to the \acrshort{DAC} conversion extrema ($\pm V_\text{ref}$) to achieve this precision. This is desirable, since it will not always be possible to achieve optimal gain due to the variable optical set-up. The actual concentration extraction is done on a computer, because of the limited computational power of the microcontroller.
  
  \begin{figure*}[!ht]
    \centering
    \begin{tikzpicture}[every node/.style={outer sep=0,inner sep=0.1cm,scale=0.75},scale=0.75]
      \node[block,design] (DAC) {\acrshort{DAC}};
      \node[block,align=center,design,right=0.75cm of DAC] (current) {$V\rightarrow I$\\conv.};
      \node[block,right=0.75cm of current,align=center] (laser) {laser};
      \node[block,right=0.75cm of laser] (gas cell) {gas cell};
      \node[block,right=0.75cm of gas cell] (PD) {\acrshort{PD}};
      \node[block,below=1.5cm of PD,align=center,design] (TIA) {\acrshort{TIA}};
      \node[left=0.75cm of TIA,minimum height=1cm,minimum width=1cm,design] (filter1) {};
      \pic[left=1.25cm of TIA] {highpass};
      \node[block,left=0.75cm of filter1,design] (VGA) {\acrshort{VGA}};
      \node[left=0.75cm of VGA,minimum height=1cm,minimum width=1cm,design] (filter2) {};
      \pic[left=1.25cm of VGA] {lowpass};
      \node[block,left=0.75cm of filter2,align=center,design] (ADC) {\SI{16}{bit}\\\acrshort{ADC}};
      \node[coordinate,left=1.25cm of ADC] (h1) {};
      \node[block,above left=0cm and 0.75cm of ADC] (LIA1) {\acrshort{1f} \acrshort{LIA}};
      \node[block,below left=0cm and 0.75cm of ADC] (LIA2) {\acrshort{2f} \acrshort{LIA}};
      \node[block,left=0.75cm of LIA2,fill=red,fill opacity=0.1,text opacity=1] (R) {PC};
      \node[below left=0cm and 0cm of LIA2] (H1) {};
      \node[above right=0cm and 0cm of LIA1] (H2) {};
      \node[fit={(H1) (H2)},draw,dashed,fill=blue,rectangle,inner sep=0cm,fill opacity=0.1] (pic) {};
      \node[below=0cm of pic,anchor=north] {microcontroller};
      \draw[line>] (DAC) -- (current);
      \draw[line>] (current) -- (laser);
      \draw[line>] (laser) -- (gas cell);
      \draw[line>] (gas cell) -- (PD);
      \draw[line>] (PD) -- (TIA);
      \draw[line>] (TIA) -- (filter1);
      \draw[line>] (filter1) -- (VGA);
      \draw[line>] (VGA) -- (filter2);
      \draw[line>] (filter2) -- (ADC);
      \draw[line>] (h1) -| (LIA1);
      \draw[line>] (h1) -| (LIA2);
      \draw[line] (ADC) -- (h1);
      \draw[line>] (LIA1) -| (R);
      \draw[line>] (LIA2) -- (R);
      \draw[line>] let \p1=(pic.north), \p2=(LIA1) in (\x2,\y1) |- (DAC);
    \end{tikzpicture}
    \caption{System architecture block diagram.}
    \label{system:architecture_block_diagram}
  \end{figure*}
  
  The architecture from \cref{system:architecture_block_diagram} provides a good trade-off between complexity and flexibility. The optical setup can easily be varied for different lasers and gasses. Emitted laser powers can vary drastically. The wide range of optical signal powers that are incident on the \acrshort{PD} need not result in a low amplitude at the \acrshort{ADC}. The \acrshort{VGA} can further amplify the signal to suitable levels for \acrshort{ADC} conversion. The use of filters (especially the high-pass filter after the \acrshort{TIA}), allows to bias the laser for different absorption peaks without needing to change the detector, since the \acrshort{DC}~bias value and triangle signal are filtered out. The input signals can also be controlled easily (from the microcontroller), making for instance an investigation into the number of bits needed for accurate measurement possible. 
  
  Lasers typically required cooling to obtain accurate output wavelengths, a \acrfull{TEC} is needed for this. This component senses a thermistor that is integrated into the laser and cools the laser down or heats it up by means of the thermo-electric effect (also called Peltier effect). A variable current is thus passed through the system. In this first prototype design, we will not yet design a custom circuit for this, but use a commercial device from Thorlabs \cite{TEC}. 
  
  \subsection{Alternative Architectures}
  
    The architecture from \cref{system:architecture_block_diagram} is not the only one. Many possible variations can be made. If the \acrshort{DC}~bias current and triangle wave amplitude coming from the \acrshort{PD} is not too large compared to the sine wave amplitude, we can omit the low pass filter. Additionally, if the optical system is fixed, the \acrshort{VGA} can also be replaced by a fixed gain amplifier or the \acrshort{TIA} can provide more gain. 
    
    However, we used the system from \cref{system:architecture_block_diagram} because if offers flexibility and can easily be adopted for different optical settings with variable attenuation and different gasses of interest, since these will also have different properties.
    
    \Cref{system:alternative_architecture_block_diagram} depicts a very different topology: the first and second order harmonic voltage signals are separated by means of band pass filters before the analogue to digital conversion step. Using this type of topology, we can optimise gain for both first and second order harmonics. 
    
    Several problems can arise with this architecture. Jitter, random differences in sample conversion timing, can be different for different analogue to digital converters (assuming these are external). Furthermore, the gain for the first and second order harmonics is no longer equal. There will be gain imbalance, even if the two \acrshortpl{VGA} are set to the same gain due to component tolerances. This means that the fitting method will also have to take the gain difference into account (see \cref{DCP}). The two harmonics of interest are also spaced quite close together (\SI{10}{\kilo\hertz} in the realised system, see \cref{laser_driver}). This will make analogue (active) filtering difficult. Finally, the overall system cost and complexity are increased (more complicated filtering, fitting and the need for two \acrshortpl{ADC}) compared to the architecture in \cref{system:architecture_block_diagram}.
  
    \begin{figure*}[!ht]
      \centering
      \begin{tikzpicture}[every node/.style={outer sep=0,inner sep=0.1cm,scale=0.9},scale=0.9]
        \node[block,design] (DAC) {\acrshort{DAC}};
        \node[block,align=center,design,right=0.75cm of DAC] (current) {$V\rightarrow I$\\conv.};
        \node[block,right=0.75cm of current,align=center] (laser) {laser};
        \node[block,right=0.75cm of laser] (gas cell) {gas cell};
        \node[block,right=0.75cm of gas cell] (PD) {\acrshort{PD}};
        \node[block,below=1.5cm of PD,align=center,design] (TIA) {\acrshort{TIA}};
        \node[coordinate,left=0.5cm of TIA] (h1) {};
        \node[block,above left=0.25cm and 0.75cm of h1,design,minimum height=1cm,minimum width=1cm] (filter 1f) {};
        \node[block,below left=0.25cm and 0.75cm of h1,design,minimum height=1cm,minimum width=1cm] (filter 2f) {};
        \pic[above left=0.75cm and 1.25cm of h1] {bandpass};
        \pic[below left=0.75cm and 1.25cm of h1] {bandpass};
        \node[block,left=0.75cm of filter 1f,minimum height=1cm,minimum width=1cm,design] (VGA 1f) {\acrshort{VGA}};
        \node[block,left=0.75cm of filter 2f,minimum height=1cm,minimum width=1cm,design] (VGA 2f) {\acrshort{VGA}};
        \node[block,left=0.75cm of VGA 1f,align=center,design] (ADC 1f) {\SI{16}{bit}\\\acrshort{ADC}};
        \node[block,left=0.75cm of VGA 2f,align=center,design] (ADC 2f) {\SI{16}{bit}\\\acrshort{ADC}};
        \node[block,left=0.75cm of ADC 1f] (LIA 1f) {\acrshort{1f} \acrshort{LIA}};
        \node[block,left=0.75cm of ADC 2f] (LIA 2f) {\acrshort{2f} \acrshort{LIA}};
        \node[block,left=0.75cm of LIA 2f,fill=red,fill opacity=0.1,text opacity=1] (R) {PC};
        \node[below left=0cm and 0cm of LIA 2f] (H1) {};
        \node[above right=0cm and 0cm of LIA 1f] (H2) {};
        \node[fit={(H1) (H2)},draw,dashed,fill=blue,rectangle,inner sep=0cm,fill opacity=0.1] (pic) {};
        \node[below=0cm of pic,anchor=north] {microcontroller};
        \draw[line>] (DAC) -- (current);
        \draw[line>] (current) -- (laser);
        \draw[line>] (laser) -- (gas cell);
        \draw[line>] (gas cell) -- (PD);
        \draw[line>] (PD) -- (TIA);
        \draw[line] (TIA) -- (h1);
        \draw[line>] (h1) |- (filter 1f);
        \draw[line>] (h1) |- (filter 2f);
        \draw[line>] (filter 1f) -- (VGA 1f);
        \draw[line>] (filter 2f) -- (VGA 2f);
        \draw[line>] (VGA 1f) -- (ADC 1f);
        \draw[line>] (VGA 2f) -- (ADC 2f);
        \draw[line>] (ADC 1f) -- (LIA 1f);
        \draw[line>] (ADC 2f) -- (LIA 2f);
        \draw[line>] (LIA 1f) -| (R);
        \draw[line>] (LIA 2f) -- (R);
        \draw[line>] let \p1=(pic.north), \p2=(LIA 2f) in (\x2,\y1) -- (DAC);
      \end{tikzpicture}
      \caption{Alternative system architecture block diagram.}
      \label{system:alternative_architecture_block_diagram}
    \end{figure*}

\end{document}
