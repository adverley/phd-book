% !TEX TS-program = xelatex
% !TEX encoding = UTF-8 Unicode


% TODO: ik heb een goed idee waar het naar toe moet hoor, tis gewoon de uitvoering en flow dat moet kloppen. Ik ben van plan het eerst eens uit te werken voor een bepaald onderdeel van het literatuur hoofdstuk, te reviewen met jou en dan die stijl\flow\template toepassen voor de andere secties in dat hoofdstuk

\providecommand{\home}{../..}
\documentclass[\home/main.tex]{subfiles}

\begin{document}

\chapter{Background and review of related work} \label{ch:lit}

The goal of the following chapter is to provide the preliminaries and a review of relevant work in the field of robotic manipulation of deformable objects. To provide some historical context, we first discuss how \emph{standard robotic manipulation pipelines} can be used for manipulating deformable objects in Section~\ref{sec:lit_traditional}. Next, in Section~\ref{sec:lit_learning} we introduce how the inherent limitations of engineered motor control architectures can be overcome by using \emph{learning-based methods}. We break this section down into subsections introducing supervised learning, deep neural networks and reinforcement learning. This is followed by surveying their applications in recent robotic manipulation work. Given the general property that learning-based methods are data-hungry, we continue this discussion by reviewing the role of \emph{large datasets} for robotic learning in Section~\ref{sec:lit_datasets}. An alternative approach to generating data is to use synthetic data. To this end, we discuss the role of \emph{simulation} and the corresponding transferability problems in Section~\ref{sec:lit_simulation}. Critical to robotic learning of manipulation skills is some metric of task success, generally labelled as reward function. The role and methods to obtain \emph{reward functions} for robotic learning, and deformable objects manipulation in particular, is reviewed in Section~\ref{sec:lit_reward_learning}. Finally, we discuss the idea and corresponding literature of \emph{instrumenting the process with sensors} to facilitate the learning process in the manipulation environment in Section~\ref{sec:lit_instrumentation}.


\section{Manipulating deformable objects} \label{sec:lit_traditional}
% TODO: mogelijks is dit hoofdstuk zo groot geworden dat het kan helpen om subsecties te introduceren. Ik denk ook om de opbouw 1D -> 2D + robotic folding -> 3D te veranderen naar: 1D -> 2d -> 3D -> robotic folding -> conclusion
% traditionele pipelines, algemeen en bondig. 

% TODO: subsections vermelden.
First we discuss the traditional approach rigid object manipulation for robots and where they fail for deformable object. Next, we provide a definition and categorization of deformable objects. For each of the categories, we provide common tasks and solutions identified in literature. Finally, we highlight historical important work in the domain of robotic folding. 

\subsection{Manipulating rigid objects}
Grasping and manipulation problems in robotics are traditionally solved by manually engineering subsystems for perception, planning and control~\autocite{Siciliano2008}. A general approach is to use images as input observation for controlling the motions of the robot. This is motivated by the advantage that images enable closed-loop control: non-contact and real-time measurements of the environment can be used to provide feedback to the motion trajectory of the robot. This principle is generally known as visual servoing~\autocite{Hutchinson1996} and was first introduced by~\citeauthor{Hill1979} in~\citeyear{Hill1979}. An archetypical pipeline consists of the following steps to grasp and manipulate an object~\autocite{Corke1996}. First, observations such as images are used to estimate the state of the object. This state estimation stage usually executes pixel manipulations and image filtering in order to extract features. This object state is in turn used for interpreting the scene in order to calculate the relative position of the target object from the robot end-effector. Once the object is identified, it can be modelled such that a suitable grasping point can be identified. Next, these grasping points are given to a motion planning system that estimates a trajectory to move the end-effector to the desired position and orientation. Finally, a low-level controller sends motor commands to the actuators to move the robot.
% TODO: add figure of some typical modular pipeline

% Probleem met traditionele pipelines toepassen voor vervormbare objecten
Engineering modular, hand-tuned motor control pipelines have been successful for applications in manufacturing~\autocite{Clocksin1985,Mochizuki1987}, car steering~\autocite{Dickmanns1988}, robotic ping-pong~\autocite{Andersson1987}, juggling~\autocite{Rizzi1993} as well as fruit picking~\autocite{Harrell1989}. However, all of these applications operate under the condition of rigid objects: the shape of the object will not change on contact. When manipulating objects, this is of importance for determining stable grasping points. More concretely, restraining rigid objects relies on \textit{form closure}~\autocite{Nguyen1988} or \textit{force closure}~\autocite{Bicchi1995}: fully constraining relative motion of the object or having contact points that can counteract an external wrench through friction. % TODO optional: foto toevoegen: form closure: portefuille volledig vastgrijpen (zie robotics springer bible pagina 1023, fig 38.6 en fig 38.8)
In the case of deformable objects, the object can deform during grasping and manipulation. This leads to exponentially higher dimensional configuration spaces~\autocite{Foresti2004} compared to rigid object manipulation. For example, achieving form closure becomes impossible as it requires immobilizing every degree of freedom. Similarly, force closure becomes computationally intractable as it requires constantly incorporating the adapted shape of the object. Furthermore, manipulation requires reasoning about the target shape of the object. These properties make the vast amount of research done in the rigid object manipulation domain inapplicable for the deformable object domain. Unfortunately, to date, the vast majority of robotic manipulation work deals with rigid objects whereas many objects are of deformable nature~\autocite{Siciliano2008}.

\subsection{Deformable objects: definition, categorization, tasks and solutions}
% Vervormbare objecten: wat zijn ze, categorisatie, welke taken en welke oude pipelines bestaan er
A deformable object is an object whose shape changes when being subject to an external force. This deformation can be temporary and reversible (\textit{elastic}), permanent (\textit{plastic}) or a combination of both (\textit{elasto-plastic}). Deformable objects are found both in industrial settings and household items. A common categorization~\autocite{Saadat2002,Jimenez2012} is based on the geometry of the object: how many dimensions are prevalent over the other dimensions. In its simplest setting, the deformable object is one dimensional: ropes, strings, cables, threads and catheters among others. These objects are also known as deformable linear objects. The term \textit{linear} refers to one dimension being dominant over the other two dimensions. Common tasks for deformable linear objects are grasping, manipulation, sensing and knot tying. Early motor control architectures for solving tasks regarding deformable linear objects used either an open-loop approach or simple visual servoing to execute the motion. An early work demonstrating modular motor control pipelines clearly is the project of~\citeauthor{Inaba1987} in~\citeyear{Inaba1987}. In their work, they use visual servoing for manipulating a rope into a ring and then tying the rope. Their perception module uses stereo images to detect the rope and the ring. The planning module is hard-coded to iterate through a set of predefined steps while using the detected center of the ring and 3D coordinates of the endpoints of the rope from the perception module. An inverse kinematics module provides a target trajectory to the low-level controller. Similar modular pipelines can be found in~\autocite{Remde1999} for grasping a rope and in~\autocite{Saha2007} where knots are tied with needles using probabilistic trajectories of the rope from a simulated model. Incorporating motion primitives, i.e. a predefined set of motor actions corresponding to high-level actions, in the planning module is used in~\autocite{Yamakawa2008, Vinh2012} to tie knots in a rope.

% TODO: add figure of some DLO's
Convergeer hier of op het einde van de sectie to the following Problems:
\begin{itemize}
    \item  open-loop behavior
    \item  simplification of the object configuration
    \item  computational expensive modeling
    \item  limited sensing for state perception(only visual)
\end{itemize}

% 2D objects klassieke pipelines
Deformable \textit{linear} objects become deformable \textit{planar} objects when two dimensions are significantly larger than the third dimension. In this case, the planning module can disregard the thickness of the material for manipulation. Canonical examples of such objects are clothing, thin-shelled objects like plastic bottles, fabric, paper, paper and plastic bags, deformable sheets, cards and foam materials. In context of this thesis, it is of interest to note that clothing satisfy the same geometrical property of this classification as other objects such as paper and plastic bottles. However, the main characteristic distinguishing clothing is the compression strength: clothing compared to other two-dimensional deformable objects do not possess any significant compression strength. Given that the current work deals with manipulations of clothing, we focus in particular on cloth folding pipelines.
% TODO: add figure of some planar deformable objects

% 3D objects klassieke pipelines
The final category of deformable objects are \textit{solid deformable objects} whose deformations across all dimensions of the object are of relevance. These are objects such as food, plush toys and sponges. In general, 3D deformable objects are the least researched type of deformable objects~\autocite{Sanchez2018}. An exception to this is soft tissue which is of importance for medical application. We refer the reader to the review paper by~\textcite{Taylor2016} for an overview of medical robots in surgery applications. For an overview on robotic manipulation of food products, we refer to the survey by~\textcite{Chua2003}.
% TODO: add figure of some 3D deformable objects

\subsection{Cloth folding pipelines}
%  Maar wat is het doel van deze subsectie?  = Inzicht verschaffen in hoe "klassieke" pipelines het probleem opdelen (is al geschreven :)) en hoe ze die problemen typisch aanpakken. Doel is dat lezer hieruit kan afleiden dat dat wel cool is maar: traag, error accumulation, geen knowledge reuse etc. 
% Cloth folding pipelines (houvast = Sanchez2018, Matas, Douman2016)
In the following Subsection, we highlight some important work in providing solutions for the subtasks of robotic folding. 

It is possible to distinguish multiple task when considering cloth manipulation applications: sensing, grasping, manipulation and cloth-specific tasks such as folding, cloth hanging and bedsheet folding. A complete cloth folding pipeline typically consists of the following subtasks: (1) grasping an isolated garment, (2) bringing it into a folded configuration and (3) stacking it on top of other folded garments. The second step in this process is often subdivided into unfolding, flattening and folding. Most of the work in robotic folding deal with a subtask instead of the complete pipeline. A notable exception to this is the work of~\textcite{Doumanoglou2016,Maitin2010} that consider the whole robotic folding pipeline.

The easiest solution for cloth folding is to use specialized hardware in a constrained environment.~\Textcite{Nair2013} propose an actuated flipfold\footnote{A flipfold is a device consisting of four panels joined by hinges. The four panels that lineup with the two sleeves, topcenter and topbottom of the shirt respectively. The hinges allow the panels to individually fold to the center panel.} that automates folding of shirts. More complex commercially available products exists such as the FoldiMate\footnote{\url{https://foldimate.com/}}. However, such products do not generalize towards general cloth folding, do not leverage general-purpose robotic hardware and have proven difficult to bring commercially available\footnote{FoldiMate has been in prototype development for nine years at time of writing.}. This is why most research consider the use of general-purpose robot arms, possibly with dedicated gripper and instrumentation, as elaborated in Section~\ref{sec:lit_instrumentation}. 

Much of the literature around cloth folding has resolved around solving subtasks of the folding pipeline. In the following paragraphs, we provide a summary of important work concerning each of the subtasks and ultimately describe two important works that consider the whole folding pipeline.

\paragraph{Grasping}
	~\autocite{Ramisa2012} grasp shirts via the collar. This is done using scale-invariant feature transform features on depth data. Their method achieves a grasping success rate of $70\%$. However, the performance drops to $30\%$ when other types of garments are present on the table.

	~\autocite{Monso2012} separate all clothing articles with a robot manipulator. They train a \acrshort{POMDP} to model uncertainty in state estimation of cloth due to occlusion.


\paragraph{Pose estimation}
	After grasping a clothing article, pose estimation is usually done such that the type and configuration of the cloth can be brought into an unfolded state, ready for folding. Garment pose estimation has been done by matching video images to simulation models~\autocite{Kita2002}, using machine learning models~\autocite{Li2014, li2014volum} or instrumentation via fiducial markers~\autocite{Bersch2011}.

    \paragraph{UNFOLDING}
	A general approach to unfold clothing articles is to exploit gravity; by grasping the article at strategic points, gravity will remove arbritary folds. ~\Textcite{Hamajima1998} exploit this gravitational trick by regrasping the hemlines of the garments. The hemlines are detected using the shadows and shape of the cloth. 

	\textcite{Cusumano2011} unfolds shirts and trousers in a two-staged pipeline using hidden markov models, a cloth simulator and a planning algorithm. By inputting the clothing article type, size and grasping points for the gripper to the \acrshort{HMM}, it can estimate the garment's configuration. This configuration is used by the simulation model to find the minimum-energy configuration. This is the configuration in which the garment's triangulated mesh vertices have minumum gravitational potential energy. Then, the planning module repeatidly executes trajectories to regrasp the clothing article until it is in a known configuration. Then,the planning module brings the garment into the hard-coded, unfolded configuration. Their method achieves a $66\%$ success rate. 

	Doumanoglou(2014) solves the same task but reduces the amount of software modules by repeatedly regrasping the lowest hanging point of the garment. This brings the clothing item into a known configuration. Next, the robot unfolds the article by searching two grasping points using a \acrshort{POMDP}. 

    \paragraph{FLATTENING}
	Before folding the cloth into the desired configuration, it is necessary to remove wrinkles caused by unfolding the cloth. Moreover, folding often relies on template matching which is made more difficult when there are wrinkles present.

	\autocite{Sun2015} propose a flattening method. Their method assume the clothing item is unfolded on a table. They employ RGB-D data to find wrinkles and represent them as fifth-order polymonials. The largest wrinkle is then flattened by using preprogrammed motions of the arms. 

	In ~\autocite{Willimon2011}, a washcloth is flattened in two phases. In the first phase, they iteratively pull the cloth away from or towards its centroid to remove minor wrinkles. The second phase utilizes depth information to determine regions of interests and direction for removing remaining wrinkles.


    \paragraph{FOLDING}
	\autocite{Bersch2011} uses fiducial markers on a piece of cloth for accurate state estimation. Then they employ an open-loop motor control to do the folding: grasp the garment by the shoulders, rotate the sleeves inwards and fold the shirt inwards while placing it on the table. 

	\autocite{Berg2010} employs a geometry based folding method: fold over predefined lines. Their method relies on using gravity to immobilize parts of the garment such that parts of the cloth become rigid objects. 

	\autocite{Miller2012} recognize the pose of the garment by fiting a user-specified polygon representation to the detected cloth contours. Then, they apply the same method as [28] to fold the cloth.

	\autocite{Yamakawa2011} start folding a cloth in midair, held by its corners, with an algebraic representation of the cloth. They use this simulation model to estimate the pose of the end-effectors at exact intervals such that open-loop trajectory of the points describes a folded garment. 

	Instead of utilizing a dual-arm robot setup,~\autocite{Petrik2017} considers folding with a single robotic arm. They compute a trajectory in simulation based on the grasping location and the folding line of the garment.


    \paragraph{FULL PIPELINE}
	The first example of a complete cloth folding pipeline is the work of~\textcite{Maitin2010}. The task starts from an unorganized pile of crumbled towels and ends when all articles are stacked in folded configuration on top of each other. Given that the end of their pipeline consists of executing predefined trajectories, much of their method rely on predecessor steps to bring the clothing into an exact known configuration. Their method start with color segmentation on the image to select the central clothing article. Next, the grasped towel is rotated and regrasped in order to find and grasp the corners visually. Unfolding is done by shaking and twisting, and pulling the towel taut. Finally, they run a predefined, open-loop trajectory to lay down the unfolded towel on the tabel and fold it. The pipepline takes $24$ minutes to execute with the grasp point detection phase being the largest bottleneck.

		nadeel aan hun methode: er is nog steeds open-loop: x aantal keer twisten en shacken en we gaan er van uit dat het unfolded is. Het is traag. 


	The second full robotic cloth folding implementation is engineered by~\citeauthor{Doumanoglou2016}. Their setup starts from a pile of shirts, towels and trousers. By segmenting an image from the pile on color, an isolated piece is grasped. 
	Their system achieves a throughput of six minutes per garment with $79\%$ success rate. The slowest step in their pipeline is the detection of the desired grasping points for unfolding. By repeatedly regrasping the lowest hanging point of the garment, they reduce the amount of possible cloth configuration to classify the garment shape using random forests~\autocite{Breiman2001}. The unfolding procedure is equivalent to~\autocite{Maitin2010} and requires a garment classifier, grasp point detector, and a pose estimation module. Flattening the shirt is done using a brush tool on a dedicated cloth folding gripper. Wrinkles are detected by comparing the contours with existing polygonal models of flattened cloths.  Finally, the fold is executed by polygonal matching to predefined templates. 
			

This review demonstrates a reoccurring theme: a divide-and-conquer methodology leads to a loss of information between the different stages, resulting in the accumulation of errors. \textcite{Doumanoglou2016} report difficulties when folding towels because the perception system labels them as shirts. These individual components are built in a laboratory environment with certain assumptions which are likely to be violated in an unstructured, complex environment. Inaccurate sensor readings together with deformation of the robot’s links also deteriorate the accuracy of these systems. In order for robots to be useful in unstructured environments with complex dynamics, there is a need for controllers that are able to perform robust grasp synthesis when faced with unseen conditions.


% Convergeren naar het besluit: al deze pipelines, gaande van 1D naar 3D deformable objects share a common divide-and-conquer approach which lead to a plethora of problems (zie lijst hierboven). Voeg ook dingen toe van FWO voorstel p10 en geef opening voor smart controllers. 

Ultimately, the motor control architectures discussed in this section suffer ...


% Dit hoort volgens mij eerder thuis in introductie:
%For example, a recent cloth folding pipeline by~\citeauthor{Doumanoglou2016} starts with a visual perception to detect candidate grasp points. These grasp points are then given to a planning module to move the robot end-effector to the desired position and orientation. This divide-and-conquer methodology leads to a loss of information between the different stages, resulting in the accumulation of errors. \Citeauthor{Doumanoglou2016} report difficulties when folding towels because the perception system labels them as shirts. These individual components are built in a laboratory environment with certain assumptions which are likely to be violated in an unstructured, complex environment. Inaccurate sensor readings together with deformation of the robot’s links also deteriorate the accuracy of these systems. In contrast, modern deep learning approaches try to achieve the same outcome in an end-to-end fashion. This is done by 

\section{Learning approaches to robotic manipulation} \label{sec:lit_learning}
% structuur: SL basics introduceren en dan toegepaste papers? Of eerst de ganse "knowledge background stack" en dan de standaard literatuur? best optie 1.

% Houvasten: Matthias, Jim, Gabriel 
\subsection{Supervised learning} \label{subsec:lit_sl}
\subsection{Deep neural networks} \label{subsec:lit_dnn}
\subsection{Reinforcement learning} \label{subsec:lit_rl}

Reinforcement Learning is an eminent approach for learning control policies with no user intervention. A complete review of RL is outside the scope of this thesis. Therefore, we refer the reader to the standard textbook of~\citeauthor{SuttonAndBarto}.

\section{Datasets for robotic learning} \label{sec:lit_datasets}
\section{Simulation environments to accelerate learning} \label{sec:lit_simulation}
\subsection{Cloth simulation methods} \label{subsec:lit_cloth_sim}
Zie ook "Robotic manipulation and sensing of deformable objects in domestic and industrial applications: a survey" p4 voor overzicht om te introduceren maar focus op particle based methods.
\subsection{Transferring simulation results to the real world}  \label{sec:lit_sim2real}
\section{Reward learning}  \label{sec:lit_reward_learning}
\section{State perception through instrumentation} \label{sec:lit_instrumentation}

\end{document}