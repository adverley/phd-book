% !TEX TS-program = xelatex
% !TEX encoding = UTF-8 Unicode


% TODO: ik heb een goed idee waar het naar toe moet hoor, tis gewoon de uitvoering en flow dat moet kloppen. Ik ben van plan het eerst eens uit te werken voor een bepaald onderdeel van het literatuur hoofdstuk, te reviewen met jou en dan die stijl\flow\template toepassen voor de andere secties in dat hoofdstuk

\providecommand{\home}{../..}
\documentclass[\home/main.tex]{subfiles}

\begin{document}

\chapter{Background and related work}\label{ch:lit} 

The goal of the following chapter is to provide the preliminaries and a review of relevant work in the field of robotic manipulation of deformable objects. We first discuss how \emph{standard robotic manipulation pipelines} can be used for manipulating deformable objects in Section~\ref{sec:lit_traditional}. Next, in Section~\ref{sec:lit_learning} we introduce how the inherent limitations of engineered motor control architectures can be overcome by using \emph{learning-based methods}. We break this section down into subsections introducing supervised learning, deep neural networks and reinforcement learning. This is followed by surveying their applications in recent robotic manipulation work. Given the general property that learning-based methods are data-hungry, we continue this discussion by reviewing the role of \emph{large datasets} for robotic learning in Section~\ref{sec:lit_datasets}. An alternative approach to generating data is to use synthetic data. To this end, we discuss the role of \emph{simulation} and the corresponding transferability problems in Section~\ref{sec:lit_simulation}. Critical to robotic learning of manipulation skills is some metric of task success, generally labelled as reward function. The role and methods to obtain \emph{reward functions} for robotic learning, and deformable objects manipulation in particular, is reviewed in Section~\ref{sec:lit_reward_learning}. Finally, we discuss the idea and corresponding literature of \emph{instrumenting the process with sensors} to facilitate the learning process in the manipulation environment in Section~\ref{sec:lit_instrumentation}.


\section{Manipulating deformable objects} \label{sec:lit_traditional}
% traditionele pipelines, bondig. 

Grasping and manipulation problems in robotics are traditionally solved by manually engineering subsystems for perception, planning and control~\autocite{Siciliano2008}. A general approach is to use images for controlling the motions of the robot. Using images enables closed-loop control: non-contact and real-time measurements of the environment can be used to provide feedback to the motion trajectory of the robot. This principle is generally known as visual servoing~\autocite{Hutchinson1996} and was first introduced by~\citeauthor{Hill1979} in~\citeyear{Hill1979}. An archetypical pipeline consists of the following steps to grasp and manipulate an object~\autocite{Corke1996}. First, observations such as images are used to estimate the state of the object. This state estimation stage usually executes pixel manipulations and image filtering in order to extract features. This object state is in turn used for interpretating the scene in order to calculate the relative position of the target object from the robot end-effector. Once the object is identified, it can be modelled such that a suitable grasping point can be identified. Next, these grasping points are given to a motion planning system that estimates a trajectory to move the end-effector to the desired position and orientation. Finally, a low-level controller sends motor commands to the actuators to move the robot. 

Engineering modular, hand-tuned motor control pipelines has been succesful for applications in manufacturing~\autocite{Clocksin1985,Mochizuki1987}, car steering~\autocite{Dickmanns1988}, robotic ping-pong~\autocite{Andersson1987}, juggling~\autocite{Rizzi1993} as well as fruit picking~\autocite{Harrell1989}. However, all of these applications operate under the condition of rigid objects: the shape of the object will not change on contact. When manipulating objects, this is of importance for determining stable grasping points. In the case of deformable objects, the object can deform during grasping and manipulation. This leads to exponantionally higher dimensional configuration spaces~\autocite{Foresti2004} compared to rigid object manipulation. However, to date the vast majority of robotic manipulation work deals with rigid objects whereas many objects are of deformable nature~\autocite{Siciliano2008}.

Deformable objects are found both in industrial settings and common household items. A common categorization is made based on the dimensionality of the deformable object. In it simplest setting, the deformable object is one dimensional: ropes, strings and cables. These objects are also known as linear deformable objects. Zie matas zijn literatuur. Hier wat referenties toevoegen over klassieke pipelines dat 1d objecten manipuleren. Dit herhalen voor 2D en 3D. Dan nieuwe paragraaf over (klassieke) cloth folding pipelines, hoe de stages typisch worden opgedeeld (zie de Griekse Andreas en Abbeel: pickup from clutter, folding) en bespreek isolating, grasping, unfolding, flattening, folding. To the best of our knowledge, only two work consider the whole pipeline: Griekse Andreas en Abbeel. Bespreek op het einde hun performance en laat de problemen naar boven komen om de nood aan learning approaches te schetsen.

Types of deformable objects. Simplest: 1D ropes, strings, cables. More complex: planar (fabric, plasti sheets). 3D: bags, clothing but also useful for surgery simulation. The work in this thesis focuses exclusively on clothing.


% Dit hoort volgens mij eerder thuis in introductie:
%For example, a recent cloth folding pipeline by~\citeauthor{Doumanoglou2016} starts with a visual perception to detect candidate grasp points. These grasp points are then given to a planning module to move the robot end-effector to the desired position and orientation. This divide-and-conquer methodology leads to a loss of information between the different stages, resulting in the accumulation of errors. \Citeauthor{Doumanoglou2016} report difficulties when folding towels because the perception system labels them as shirts. These individual components are built in a laboratory environment with certain assumptions which are likely to be violated in an unstructured, complex environment. Inaccurate sensor readings together with deformation of the robot’s links also deteriorate the accuracy of these systems. In contrast, modern deep learning approaches try to achieve the same outcome in an end-to-end fashion. This is done by 

\section{Learning approaches to robotic manipulation} \label{sec:lit_learning}
% structuur: SL basics introduceren en dan toegepaste papers? Of eerst de ganse "knowledge background stack" en dan de standaard literatuur? best optie 1.

\subsection{Supervised learning} \label{subsec:lit_sl}
\subsection{Deep neural networks} \label{subsec:lit_dnn}
\subsection{Reinforcement learning} \label{subsec:lit_rl}

Reinforcement Learning is an eminent approach for learning control policies with no user intervention. A complete review of RL is outside the scope of this thesis, the reader is referred to the standard textbook of~\citeauthor{SuttonAndBarto}. 

\section{Datasets for robotic learning} \label{sec:lit_datasets}
\section{Simulation environments to accelerate learning} \label{sec:lit_simulation}
\subsection{Cloth simulation methods} \label{subsec:lit_cloth_sim}
\subsection{Transferring simulation results to the real world}  \label{sec:lit_sim2real}
\section{Reward learning}  \label{sec:lit_reward_learning}
\section{State perception through instrumentation} \label{sec:lit_instrumentation}

\end{document}