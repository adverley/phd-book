% !TEX TS-program = xelatex
% !TEX encoding = UTF-8 Unicode

\providecommand{\home}{../..}
\documentclass[\home/main.tex]{subfiles}

\begin{document}

\chapter{Background and related work}\label{ch:sota} 

The following chapter provides a comprehensive review of the field of robotic manipulation of deformable objects. We first introduce the particularities that distinguish deformable objects from their rigid counterpart in Section~\ref{sec:test}. Then, we briefly discuss engineered robotic manipulation pipelines while highlighting their strengths and pitfalls. Section~\ref{sec:lit_learning} deals.

\section{Manipulating deformable objects} \label{sec:test}
% traditionele pipelines, bondig. Doel: nood aan learning approaches schetsen

\section{Learning approaches to robotic manipulation} % SL, behavioral cloning, RL 
\subsection{Supervised learning}
\subsection{Deep neural networks}
\subsection{Reinforcement learning}

\section{Datasets for robotic learning}
\section{Simulation environments to accelerate learning}
\subsection{Cloth simulation methods}
\subsection{Transferring simulation results to the real world}
\section{Reward signals to learn tasks}
\section{State perception through instrumentation}

\end{document}