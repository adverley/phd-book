\providecommand{\home}{../../..}
\documentclass[\home/main.tex]{subfiles}
% \documentclass[twoside,11pt,a4paper]{article}
% \usepackage{tikz}

\usetikzlibrary{decorations.pathreplacing}
\usetikzlibrary{fadings}

\begin{document}

% from https://tex.stackexchange.com/a/375862

\def\layersep{1.5cm}


\begin{tikzpicture}[
   shorten >=1pt,->,
   draw=black!50,
    node distance=\layersep,
    every pin edge/.style={<-,shorten <=1pt},
    neuron/.style={circle,fill=black!25,minimum size=17pt,inner sep=0pt},
    input neuron/.style={neuron, fill=ColorAccent1},
    output neuron/.style={neuron, fill=ColorAccent2},
    hidden neuron/.style={neuron, fill=ColorMainLight},
    annot/.style={text width=4em, text centered}
]
    % set number of hidden layers
    \newcommand\Ninputs{4}
    \newcommand\Nhidden{2}
    \newcommand\NhiddenNeurons{5}
    \newcommand\Noutputs{3}
    \newcommand\OffsetOutputLayer{-0.5cm}

    % Draw the input layer nodes
    \foreach \name / \y in {1,...,\Ninputs}
    % This is the same as writing \foreach \name / \y in {1/1,2/2,3/3,4/4}
        % \node[input neuron, pin=left:Input \#\y] (I-\name) at (0,-\y) {};
        \node[input neuron] (I-\name) at (0,-\y) {};

    % Draw the hidden layer nodes
    \foreach \N in {1,...,\Nhidden} {
       \foreach \y in {1,...,\NhiddenNeurons} {
          \path[yshift=0.5cm]
              node[hidden neuron] (H\N-\y) at (\N*\layersep,-\y cm) {};
           }
    \node[annot,above of=H\N-1, node distance=1cm] (hl\N) {};
    }

    % Draw the output layer node
    % \node[output neuron,pin={[pin edge={->}]right:Output}, right of=H\Nhidden-3] (O) {};
    % \node[output neuron, right of=H\Nhidden-3] (O) {};
    \foreach \y in {1,...,\Noutputs} {
        \path[yshift=\OffsetOutputLayer] node[output neuron, right of=H\Nhidden-3] (OUT-\y) at (\Nhidden*\layersep,-\y cm) {};
                }


    % Connect every node in the input layer with every node in the hidden layer.
    \foreach \source in {1,...,\Ninputs}
        \foreach \dest in {1,...,\NhiddenNeurons}
            \path (I-\source) edge (H1-\dest);

    % connect all hidden stuff
    \foreach [remember=\N as \lastN (initially 1)] \N in {2,...,\Nhidden}
       \foreach \source in {1,...,\NhiddenNeurons}
           \foreach \dest in {1,...,\NhiddenNeurons}
               \path (H\lastN-\source) edge (H\N-\dest);

    % Connect every node in the hidden layer with the output layer
    \foreach \source in {1,...,\NhiddenNeurons}
        \foreach \dest in {1,...,\Noutputs}
            \path (H\Nhidden-\source) edge (OUT-\dest);

    % Annotate the layers
    % \node[annot,left of=hl1] {Inputs};
    % \node[annot,right of=hl\Nhidden] {Outputs};

\end{tikzpicture}

\end{document}
