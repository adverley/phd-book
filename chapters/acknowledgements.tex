% !TEX TS-program = xelatex
% !TEX encoding = UTF-8 Unicode

\providecommand{\home}{..}
\documentclass[\home/main.tex]{subfiles}

\begin{document}
	
\chapter{Dankwoord}

% INTRO 
Economie studeren, optimalisatiesoftware ontwikkelen en dan een doctoraat doen in de robotica en AI. 't Is geen evidentie. Het is een hele onderneming waarbij de ondersteunde groep mensen om je heen het grootste deel van je ervaring bepalen. 
Ik ben een geluksvogel. Doctoreren is zo een intellectuele vrijheid. Ik heb daarbij het geluk een bijzondere groep personen rond mij te hebben die dit verhaal mee helpen waarmaken. 

Mijn eerste dankbetuiging gaat uit naar mijn hoofdpromotor prof. Francis wyffels. Francis, jouw technische inzichten aan dit doctoraat zijn vanzelfsprekend. Waar ik jou in het bijzonder dankbaar voor ben, is jouw toewijding om voor iedere doctoraatstudent opnieuw een stukje van jezelf te investeren zodat een doctoraatstraject tot een goed eind komt. Meer concreet, ik waardeer het aanspreken van je persoonlijk netwerk om o.a.\ de vouwtafel te bouwen, mee op pad te gaan om kandidaten te ronselen voor de was te vouwen en de oneindig vele kansen. Ik waardeer jouw toezicht op de vooruitgang van mijn onderzoek en de bijzondere aandacht voor mijn persoonlijk welzijn. Het hele team geniet deze privileges en dit maakt het uitzonderlijk aangenaam werken in deze omgeving. Deze goede omgeving is ook te danken aan de twee andere AIRO hoofden, prof. Tony Belpaeme en mijn copromotor prof. Joni Dambre. Joni, bedankt om je vleugels uit te slagen over alle mensen in de groep. Ook bedankt voor de aangename en leerzame babbels. In dit opzicht gaat mijn waardering ook uit naar alle juryleden om dit proefschrift kritisch en met volle aandacht onder handen te nemen.
Uiteraard kon dit alles niet zonder de nodige financiële en administratieve steun die ik kunnen genieten hebben. Hiervoor ben ik dank verschuldigd aan Piet Demeester, de Vlaamse overheid en de administratieve ondersteuning vanuit onze vakgroep en de UGent.
% TODO: afstemmen met Francis: mag dit geweten zijn? 

Het 8e verdiep in iGent, de AIRO groep, is vijf jaar mijn thuis geweest. 

Tom, mijn eerste jaren in het lab hadden zeer eenzaam geweest zonder jou in het robotlabo. Olivier, bedankt om onze haag te helpen planten, de gezellige babbels en alle \LaTeX \textit{wizardry}!  
Matthijs, bedankt voor de bijzondere implementatiehulp en denkwerk met de TCN's. Jouw bijdragen zijn onmisbaar geweest. Victor, het is gevaarlijk om ons samen te zetten want dan brainstormen we de hele dag! 
In volgorde van ancienniteit: Matthias, Mathieu, Maxime, Pieter, Rembert, Natacha, Peter, Zimcke 
Een aantal AIRO alumni: Lio, Dries, Alexander, Gabriel, Jonas, Ira, Jeroen, Fréderic. Ik ben blij de dynamiek van de vorige generatie nog te mogen meemaken. 
De nieuwe generatie, Thomas, Remko en Dries, jullie zijn een vat vol talent! Ik kijk uit naar jullie toekomstige publicaties. 


% Family and friends
Ouders en familie 
David, tijdens de blok speelden we UT04, tijdens mijn doctoraat Tom Clancy en Divinity II, het heeft me deugd gedaan. 
Dichte vriendenkring. Rik Wieland Jeroen Thomas Tim. De geregelde ontluchting en jaarlijkse detox door een weekje te gamen in de ardennen is iets wat ik aan weinig mensen uitgelegd krijgt. Dit is wat onze vriendschap zo speciaal maakt. 
Maura en Tom
Sam en Eva 
Sam, jij hebt in 2013 initiatief genomen om in contact te komen met de groep dat nu vijf jaar mijn thuis is geweest, bedankt!
Qliff, Storm, jullie zijn de twee beste en meest harige collega's dat ik thuis had tijdens de lockdown van de COVID-19 pandemie. Tot slot, Stefanie 

\vspace{1cm}

\begin{flushright}
\textit{\theauthor, \today}
\end{flushright}



\end{document}