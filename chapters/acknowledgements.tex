% !TEX TS-program = xelatex
% !TEX encoding = UTF-8 Unicode

\providecommand{\home}{..}
\documentclass[\home/main.tex]{subfiles}

\begin{document}

\chapter{Dankwoord}

% INTRO: hook
Dit boek is meer dan de som van gepubliceerde hoofdstukken in wetenschappelijke tijdschriften. Het is de accumulatie van hard werk en volharding dat niet mogelijk zou geweest zijn zonder de ondersteuning van vele anderen. Ik ben een geluksvogel. Een bijzondere groep personen rond mij schrijft mee aan dit verhaal. Ik wens dan ook iedereen die betrokken was bij mijn onderzoek, zowel op professioneel als op persoonlijk vlak, daarom van harte te bedanken. In het bijzonder zij die hier niet bij naam genoemd zijn.

% Promotoren en jury
Mijn eerste dankbetuiging gaat uit naar mijn hoofdpromotor prof. Francis wyffels.
Francis, jouw technische bijdragen aan dit doctoraat zijn vanzelfsprekend. Ik ben je in het bijzonder dankbaar voor jouw toewijding om voor iedere doctoraatstudent een stukje van jezelf te investeren zodat onze doctoraatstrajecten tot een goed einde komen.
Jouw hulp ging veel verder dan wekelijkse feedbackmomenten; je ging mee op pad om kandidaten kledijvouwers te ronselen, bent 24/7 bereikbaar, helpt mee de grote lijnen en kleine onderzoeksdetails uit te pluizen, biedt oneindig vele kansen en waakt over ons mentaal welzijn.
Ons hele team geniet van deze privileges en dit maakt ons onderzoekslabo een aangename werkomgeving. Deze omgeving is ook te danken aan de twee andere AIRO hoofden, mijn copromotor prof.~Joni Dambre en prof.~Tony Belpaeme. Joni, bedankt voor de leerzame babbels. In dit opzicht gaat mijn waardering ook uit naar alle juryleden om dit proefschrift kritisch en met volle aandacht onder handen te nemen.
Uiteraard kon dit alles niet zonder de nodige financiële en administratieve steun die ik kunnen genieten hebben. Hiervoor ben ik dank verschuldigd aan de Vlaamse overheid en de administratieve ondersteuning van onze vakgroep en de UGent.

% op het werk, collegas algemeen
De achtste verdieping in iGent, de AIRO onderzoeksgroep, is vijf jaar mijn thuis geweest. De dynamische en intellectueel uitdagende sfeer aan onze onderzoeksgroep is te danken aan zowel alumni: Lio, Dries, Alexander, Gabriel, Jonas, Ira, Jeroen, Fréderic en Luthffi, als aan de huidige collega's: Tom, Olivier, Matthijs, Victor, Thomas, Remko, Dries, Matthias, Mathieu, Maxime, Pieter, Rembert, Natacha, Peter, Zimcke, Axel, Benedikt, Bjarne, Jeanne, Maria, Qiaoqiao, Ruben, Stefan en Tanguy.

% Collegas in detail
Tom, zonder jou had mijn eerste jaar in het robotlabo maar een eenzame bedoeling geweest. Olivier, jouw gekke \LaTeX\ tovenarij heeft me vele uren tijdens het schrijven van dit doctoraat bespaard. Ook bedankt voor alle gezellige babbels.
Matthijs, jouw implementatiehulp en denkwerk zijn onmisbaar geweest voor enkele spannende resultaten in dit boek.
Victor, het is steeds een uitdaging om onze brainstorms tot een uur te beperken.
Onze nieuwe, talentvolle generatie roboticaonderzoekers; Thomas, Remko en Dries, ik kijk uit naar jullie toekomstige verwezenlijkingen.

% Family and friends
Waar ik mijn collega's te danken heb voor alle technische en wetenschappelijke hulp, is de morele ondersteuning van mijn vrienden onmisbaar geweest.
Wieland, Jeroen, Thomas, Tim en Rik, we gaan al sinds de kleuterklassen mee. Bedankt voor jullie vriendschap. De jaarlijkse detox door een week te gamen in de Ardennen is iets wat ik aan weinig mensen uitgelegd krijg.
Maura en Tom, ik kijk iedere maand opnieuw uit naar onze kookpartijen en boardgameavonden.
Sam, jij hebt me in 2013 in contact gebracht met de groep waarin ik nu dit doctoraat afrond, bedankt!
De \textit{band}: Jeroen, Lieven, Renaat, maar toch ook nog een beetje Laurens; we zijn gekend onder drie namen (momenteel als \textit{Anna Lies}, zoek ons op Spotify!) maar we zijn toch al 13 jaar dezelfde individuen en mijn grootste muzikale uitlaatklep.
Ik wil verder ook mijn ouders, mijn zus Anastasia en mijn broer David bedanken voor het warme nest waarin ik mogen opgroeien heb. 
Mama en papa, bedankt voor alle onvoorwaardelijke steun. Ik heb zorgeloos kunnen studeren en mezelf kunnen ontwikkelen tot de persoon die ik nu ben. Het is niet vanzelfsprekend maar ik heb het toch als vanzelfsprekend kunnen ervaren.
Qliff en Storm, jullie zijn de twee liefste en meest knuffelbare collega's dat ik thuis had tijdens de lockdown van de COVID-19 pandemie.

Tot slot, Stefanie, jij bent al zoveel jaar naast mijn zijde. Je was er bij toen we van de middelbare schoolbanken kwamen, we hebben samen gestudeerd, samen beginnen te werken en je was er toen ik de stap terug naar de academische wereld maakte. En je bent er nu nog steeds. 
Je geeft me een reden om 's avonds weg te stappen van het computerscherm en naar buiten te komen in onze tuin, samen te koken, onze honden te knuffelen, en de wereld rond te reizen. Je maakt van ons huis een thuis. Bedankt voor alles. 

\vspace{1cm}

\begin{flushright}
    \textit{\theauthor, \today}
\end{flushright}



\end{document}