% !TEX TS-program = xelatex
% !TEX encoding = UTF-8 Unicode

\providecommand{\home}{..}
\documentclass[\home/main.tex]{subfiles}

\begin{document}

\chapter{Dankwoord}

% INTRO: hook
Dit boek is meer dan de som van gepubliceerde hoofdstukken in wetenschappelijke tijdschriften. Het is de accumulatie van hard werk en volharding, dat niet mogelijk is zonder de ondersteuning van vele anderen. Ik ben een geluksvogel. Een bijzondere groep personen rond mij schrijft mee aan dit verhaal. Ik wens dan ook iedereen die betrokken was bij mijn onderzoek, zowel op professioneel als op persoonlijk vlak, van harte te bedanken. In het bijzonder zij die hier niet bij naam genoemd zijn.
 
% Promotoren en jury
Mijn eerste dankbetuiging gaat uit naar mijn hoofdpromotor professor Francis wyffels. Francis, jouw technische bijdragen aan dit doctoraat zijn vanzelfsprekend. Ik ben je in het bijzonder dankbaar voor jouw toewijding. Je investeert in iedere doctoraatstudent een stukje van jezelf. Jouw hulp gaat veel verder dan wekelijkse feedbackmomenten. Je ging mee op pad om kandidaten kledijvouwers te ronselen, bent 24/7 bereikbaar, helpt mee de grote lijnen en kleine onderzoeksdetails uit te pluizen, biedt oneindig vele kansen en waakt over ons mentaal welzijn. Jouw bezieling straalt af op ons hele team en maakt van het onderzoekslabo een aangename werkomgeving.
Voor die werksfeer die ik ook mijn twee andere AIRO-hoofden te bedanken: mijn copromotor professor Joni Dambre en professor Tony Belpaeme. Joni, bedankt voor de leerzame babbels. 
Mijn waardering gaat ook uit naar alle betrokken juryleden om dit proefschrift kritisch en met volle aandacht onder handen te nemen.
Uiteraard kan dit alles niet gerealiseerd worden zonder de nodige financiële en administratieve steun. Hiervoor ben ik dank verschuldigd aan de Vlaamse overheid en aan de administratieve ondersteuning van onze vakgroep en de UGent.
 
% op het werk, collegas algemeen
De achtste verdieping in iGent, de AIRO onderzoeksgroep, is vijf jaar mijn thuis geweest. Er leeft een dynamische en intellectueel uitdagende sfeer, die te danken is aan de alumni en aan de huidige collega’s: Lio, Dries, Alexander, Gabriel, Jonas, Ira, Jeroen, Fréderic, Luthffi, Tom, Olivier, Matthijs, Victor, Thomas, Remko, Dries, Matthias, Mathieu, Maxime, Pieter, Rembert, Natacha, Peter, Zimcke, Axel, Benedikt, Bjarne, Jeanne, Maria, Qiaoqiao, Ruben, Stefan en Tanguy.
 
% Collegas in detail
Tom, zonder jou had mijn eerste jaar in het robotlabo een eenzame bestaan geweest. Olivier, jouw gekke \LaTeX\ tovenarij bespaarde me vele uren tijdens het schrijven van dit doctoraat. Onze gezellige babbels waren een welgekomen afwisseling.
Matthijs, jouw implementatiehulp en denkwerk waren onmisbaar voor enkele spannende resultaten in dit boek.
Victor, het is steeds een uitdaging om onze brainstorms tot een uur te beperken.
Thomas, Remko en Dries, Ik kijk al uit naar jullie verwezenlijkingen. 
 
% Family and friends
De technische en wetenschappelijke hulp van mijn collega's was bijzonder waardevol, net zoals de morele ondersteuning van mijn vrienden.
Wieland, Jeroen, Thomas, Tim en Rik, we kennen elkaar al sinds de kleuterklas. Bedankt voor jullie vriendschap. Jaarlijks een week gamen in de Ardennen als detox is iets wat ik aan weinig mensen uitgelegd krijg.
Maura en Tom, ik kijk iedere maand opnieuw uit naar onze kookpartijen en boardgame-avonden.
Sam, jij bracht me in 2013 in contact met de groep waarin ik nu dit doctoraat afrond. Bedankt.
De \textit{band}: Jeroen, Lieven, Renaat, maar toch ook nog een beetje Laurens. We zijn gekend onder drie namen (momenteel als \textit{Anna Lies}, zoek ons op Spotify!), maar we zijn toch al dertien jaar dezelfde individuen en samen vormen jullie mijn grootste muzikale uitlaatklep.
 
Ik wil verder ook mijn ouders, mijn zus Anastasia en mijn broer David bedanken voor het warme nest waarin ik opgroeide.
Mama en papa, bedankt voor alle onvoorwaardelijke steun. Ik kon zorgeloos studeren en mezelf ontwikkelen tot de persoon die ik nu ben. Het is niet vanzelfsprekend, maar ik heb het toch altijd zo mogen ervaren.
Qliff en Storm, jullie zijn de liefste en meest knuffelbare collega's tijdens het vele thuiswerken door de COVID-19 pandemie.
 
Tot slot, Stefanie, jij staat al zoveel jaar naast mijn zijde. Je was erbij toen we van de middelbare schoolbanken kwamen, we hebben samen gestudeerd, begonnen op hetzelfde moment te werken en je steunde me voluit toen ik de stap naar de academische wereld maakte. En je bent er nog steeds. Je geeft me een reden om 's avonds het computerscherm uit te zetten en naar buiten te komen in onze tuin, samen te koken, onze honden te knuffelen en de wereld rond te reizen. Je maakt van ons huis een thuis. Bedankt voor alles.


\vspace{1cm}

\begin{flushright}
    \textit{\theauthor, \today}
\end{flushright}



\end{document}