% !TEX TS-program = xelatex
% !TEX encoding = UTF-8 Unicode

\providecommand{\home}{..}
\documentclass[\home/main.tex]{subfiles}

\begin{document}

\chapter{Dankwoord}

% INTRO: hook?
Van economie studeren naar optimalisatiesoftware ontwikkelen om uiteindelijk een doctoraat te schrijven.
Het is een heuse onderneming waarbij de ondersteunde groep mensen om mij heen het grootste deel van mijn ervaring bepaald hebben.
Ik ben een geluksvogel. Een bijzondere groep personen rond mij heeft dit verhaal mee waargemaakt.

% Promotoren en jury
Mijn eerste dankbetuiging gaat uit naar mijn hoofdpromotor prof. Francis wyffels.
Francis, jouw technische bijdragen aan dit doctoraat zijn vanzelfsprekend. Ik ben je in het bijzonder dankbaar voor jouw toewijding om voor iedere doctoraatstudent een stukje van jezelf te investeren zodat onze doctoraatstrajecten tot een goed einde komen.
Jouw hulp ging veel verder dan wekelijkse feedbackmomenten; je ging mee op pad gaan om kandidaat kledijvouwers te ronselen, bent 24/7 bereikbaar, helpt mee de grote lijnen en kleine onderzoeksdetails uit te pluizen, biedt oneindig vele kansen en waakt over onze mentale welzijn.
Ons hele team geniet van deze privileges en dit maakt ons onderzoekslabo een aangename werkomgeving. Deze omgeving is ook te danken aan de twee andere AIRO hoofden, mijn copromotor prof. Joni Dambre en prof. Tony Belpaeme. Joni, bedankt om je vleugels uit te slagen over alle mensen in de groep. Ook bedankt voor de aangename en leerzame babbels. In dit opzicht gaat mijn waardering ook uit naar alle juryleden om dit proefschrift kritisch en met volle aandacht onder handen te nemen.
Uiteraard kon dit alles niet zonder de nodige financiële en administratieve steun die ik kunnen genieten hebben. Hiervoor ben ik dank verschuldigd aan Piet Demeester, de Vlaamse overheid en de administratieve ondersteuning vanuit onze vakgroep en de UGent.

% op het werk, collegas algemeen
De achtste verdieping in iGent, de AIRO groep, is vijf jaar mijn thuis geweest. De dynamische en intellectueel uitdagende sfeer aan onze onderzoeksgroep is te danken aan zowel alumni: Lio, Dries, Alexander, Gabriel, Jonas, Ira, Jeroen, Fréderic en Luthffi, als aan de huidige collega's: Tom, Olivier, Matthijs, Victor, Thomas, Remko, Dries, Matthias, Mathieu, Maxime, Pieter, Rembert, Natacha, Peter, Zimcke, Axel, Benedikt, Bjarne, Jeanne, Maria, Qiaoqiao, Ruben, Stefan en Tanguy.

% Collegas in detail
Tom, zonder jou had mijn eerste jaar in het robotlabo maar een eenzame bedoeling geweest. Olivier, jouw gekke \LaTeX\ tovenarij heeft me vele uren tijdens het schrijven van dit doctoraat bespaard. Ook bedankt voor alle gezellige babbels.
Matthijs, jouw implementatiehulp en denkwerk zijn onmisbaar geweest voor enkele spannende resultaten in dit boek.
Victor, het is nog steeds een uitdaging om brainstorms met jou tot een uur te beperken.
De nieuwe generatie roboticaonderzoekers; Thomas, Remko en Dries, jullie zitten bomvol talent! Ik kijk uit naar jullie toekomstige verwezelijkingen.

% Family and friends
Waar ik mijn collega's te danken heb voor alle technische en wetenschappelijke hulp, is de morele ondersteuning van mijn vrienden onmisbaar geweest.
Wieland, Jeroen, Thomas, Tim en Rik, we gaan al sinds de kleuterklassen mee. Bedankt voor jullie vriendschap. De jaarlijkse detox door een week te gamen in de ardennen is iets wat ik aan weinig mensen uitgelegd krijgt.
Maura en Tom: ik kijk iedere maand opnieuw uit naar onze kookpartijen en boardgameavonden.
Sam, jij hebt me in 2013 in contact gebracht met de groep waarin ik nu dit doctoraat afrond, bedankt!
De \textit{band}: Jeroen, Lieven, Renaat, maar toch ook nog een beetje Laurens; we zijn gekend onder drie namen (momenteel als \textit{Anna Lies}, zoek ons op Spotify!) maar we zijn toch al 13 jaar dezelfde individuen en mijn grootste muzikale uitlaatklep.
David, tijdens mijn studententijd in de blok speelden we UT04, tijdens mijn doctoraat Tom Clancy en Divinity II, het heeft me deugd gedaan.
Anastasia??
Finaal kom ik dichter bij de mensen die mijn ankerpunten zijn. Mama, papa, bedankt voor alle onvoorwaardelijke steun. Ik heb zorgenloos kunnen studeren en mezelf kunnen ontwikkelen tot de persoon die ik nu ben. Het is niet vanzelfsprekend maar ik heb het toch als vanzelfsprekend kunnen ervaren.
Qliff, Storm, jullie zijn de twee liefste en meest knuffelbare collega's dat ik thuis had tijdens de lockdown van de COVID-19 pandemie.
Tot slot, Stefanie, dit dankwoord startte chronologisch in 2009, hetzelfde jaar waarop we elkaar leren kennen hebben. Ik timmer ook al aan deze afgelegde weg sinds toen.

\vspace{1cm}

\begin{flushright}
    \textit{\theauthor, \today}
\end{flushright}



\end{document}