% !TEX TS-program = xelatex
% !TEX encoding = UTF-8 Unicode

\providecommand{\home}{../..}
\documentclass[\home/main.tex]{subfiles}

\begin{document}

\chapter{Conclusion}\label{ch:conclusion}

\section{Challenges in robotic folding}

\section{Research conclusions}
Filled gaps:
    Simulation on GPU -> rapid + massive parallel 


\section{Future research directions} \label{sec:conc_future_work}
In \cref{ch:towards_robotic_folding} we elaborated on our vision for which directions the field of robotic folding should advance towards. In this section, we zoom in on specifically improvements and future directions for the research and contributions in our research. 

Utilize smart cloth of \cref{ch:instrumentation} for state representation: communicate tactile sensing or state categories. Much better than learning the state implicitly from pixels. 

% Labeled cloth: images, depth, tactile sensing 


SIMULATION:
    Tuning physical parameters. Now wonky. Differentiable simulation for estimating physical parameters.
    Massive in parallel with multiple robots. 

\end{document}
