% !TEX TS-program = xelatex
% !TEX encoding = UTF-8 Unicode

\providecommand{\home}{../..}
\documentclass[\home/main.tex]{subfiles}

\begin{document}

\chapter{Conclusion}\label{ch:conclusion}

\section{Challenges in robotic folding}

\section{Research conclusions}
Filled gaps:
    Simulation on GPU -> rapid + massive parallel 


\section{Future research directions} \label{sec:conc_future_work}
In \cref{ch:towards_robotic_folding} we elaborated on our vision for which directions the field of robotic folding should advance towards. In this section, we zoom in on specifically improvements and future directions for the research and contributions in our research. 

Utilize smart cloth of \cref{ch:instrumentation} for state representation: communicate tactile sensing or state categories. Much better than learning the state implicitly from pixels. 

% Labeled cloth: images, depth, tactile sensing 

TCN:
    multimodal embedding, bv zoals \autocite{making sense of vision and touch}: architecture backbone per modality that is fused. Of zoals https://ieeexplore.ieee.org/abstract/document/9226466 fig 5 p7 waarbij each modality ook een aparte encoder heeft en de verschillende encoders worden gecontrast. 

    Zie ook future outlook p23 van https://ieeexplore.ieee.org/abstract/document/922646
        architectuur verschillen
        future of contrastive loss

    How to generate meaningful representation when modalities are dropped is an open research question. We propose ... 


REWARD FUNCTION 
    safety, exploit environment -> problem present in RL with engineered reward function. Maybe even more present in learned reward funcitons given that neural networks are easily fooled \autocite{nn are easily fooled, adversarial examples}. To date, there is no solution for this. The problem of ensuring that a reinforcement learning agent’s goal is aligned with our own goals is unsolved \autocite{Sutton2018}. 

SIMULATION:
    Tuning physical parameters. Now wonky. Differentiable simulation for estimating physical parameters.
    Massive in parallel with multiple robots. 

\end{document}
