% !TEX TS-program = xelatex
% !TEX encoding = UTF-8 Unicode

\providecommand{\home}{../..}
\documentclass[\home/main.tex]{subfiles}

\begin{document}

\chapter{Cloth simulation parameters}\label{appx:cloth_sim_params}
Simulation parameter tuning is necessary for the cloth simulation developed in \cref{ch:simulation} to be stable and resemble cloth-like behavior. In \cref{table:cloth_sim_params}, we give the used parameters we found to be stable and useful in our simulation setup.

\begin{table}[htb]
  \centering
  \begin{threeparttable}
  \caption{The used cloth simulation physical parameters.}
  \begin{tabular}[t]{@{} l r @{}} 
    \toprule
    Parameter                           &   Value	 						\\
    \midrule
    Gravity multiplier     	            & 0.25				\\
    Cloth time subdivision     	        & 20   				\\
    Integration scheme      	          & Verlet   				\\
    Elastic spring constant             & 5400   				\\
    Shear spring constant      	        & 3000   				\\
    Bending spring constant             & 2400   				\\
    Inverse mass       	                & 0.125   				\\
    Physics damping      	              & 38   				\\
    Restition constant\tnote{*}   	    & 0.025   				\\
    Friction constant\tnote{$\dagger$}    	    & 0.95   				\\
    Elastic spring constant for meshes  & 2400   				\\
    Shear spring constant for meshes    & 2400   				\\

    \bottomrule
  \end{tabular}
  \begin{tablenotes}\footnotesize
    \item[*] How much relative velocity is kept after collision.
    \item[$\dagger$] Friction on surfaces due to collision.
  \end{tablenotes}

  \label{table:cloth_sim_params}
  \end{threeparttable}
  \end{table}

\end{document}
